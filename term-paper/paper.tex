\documentclass[aps,twocolumn,secnumarabic,balancelastpage,amsmath,amssymb,nofootinbib]{revtex4}
\usepackage{amsmath}
\usepackage{amssymb}
\usepackage{amsfonts}
\usepackage{chapterbib}
\usepackage{color}
\usepackage{graphics}
\usepackage[pdftex]{graphicx}
\usepackage{grffile}
\usepackage{longtable}
\usepackage{epsf}
\usepackage{bm}
\usepackage{tikz}
\usepackage{asymptote}
\usepackage{thumbpdf}
\usepackage{bbm}
\usepackage{amscd}
\usepackage{units}
\usepackage{textcomp}
\usepackage[utf8x]{inputenc}
\usepackage{feyn}
\usepackage{feynmp}
\usepackage[colorlinks=true]{hyperref}
\newcommand{\drelectron}[1]{\node at #1 [circle, draw, inner sep=0pt, minimum size=1pt] {\_}}

\newcommand{\ud}{\mathrm{d}}
\newcommand{\ue}{\mathrm{e}}
\newcommand{\ui}{\mathrm{i}}
\newcommand{\res}{\mathrm{Res}}
\newcommand{\Tr}{\mathrm{Tr}}
\newcommand{\dsum}{\displaystyle\sum}
\newcommand{\dprod}{\displaystyle\prod}
\newcommand{\dlim}{\displaystyle\lim}
\newcommand{\dint}{\displaystyle\int}
\newcommand{\fsno}[1]{{\!\not\!{#1}}}
\newcommand{\eqar}[1]
{
  \begin{align*}
    #1
  \end{align*}
}
\newcommand{\eqarn}[1]
{
  \begin{align}
    #1
  \end{align}
}
\newcommand{\texp}[2]{\ensuremath{{#1}\times10^{#2}}}
\newcommand{\dexp}[2]{\ensuremath{{#1}\cdot10^{#2}}}
\newcommand{\eval}[2]{{\left.{#1}\right|_{#2}}}
\newcommand{\paren}[1]{{\left({#1}\right)}}
\newcommand{\lparen}[1]{{\left({#1}\right.}}
\newcommand{\rparen}[1]{{\left.{#1}\right)}}
\newcommand{\abs}[1]{{\left|{#1}\right|}}
\newcommand{\sqr}[1]{{\left[{#1}\right]}}
\newcommand{\crly}[1]{{\left\{{#1}\right\}}}
\newcommand{\angl}[1]{{\left\langle{#1}\right\rangle}}
\newcommand{\tpdiff}[4][{}]{{\paren{\frac{\partial^{#1} {#2}}{\partial {#3}{}^{#1}}}_{#4}}}
\newcommand{\tpsdiff}[4][{}]{{\paren{\frac{\partial^{#1}}{\partial {#3}{}^{#1}}{#2}}_{#4}}}
\newcommand{\pdiff}[3][{}]{{\frac{\partial^{#1} {#2}}{\partial {#3}{}^{#1}}}}
\newcommand{\diff}[3][{}]{{\frac{\ud^{#1} {#2}}{\ud {#3}{}^{#1}}}}
\newcommand{\psdiff}[3][{}]{{\frac{\partial^{#1}}{\partial {#3}{}^{#1}} {#2}}}
\newcommand{\sdiff}[3][{}]{{\frac{\ud^{#1}}{\ud {#3}{}^{#1}} {#2}}}
\newcommand{\tpddiff}[4][{}]{{\left(\dfrac{\partial^{#1} {#2}}{\partial {#3}{}^{#1}}\right)_{#4}}}
\newcommand{\tpsddiff}[4][{}]{{\paren{\dfrac{\partial^{#1}}{\partial {#3}{}^{#1}}{#2}}_{#4}}}
\newcommand{\pddiff}[3][{}]{{\dfrac{\partial^{#1} {#2}}{\partial {#3}{}^{#1}}}}
\newcommand{\ddiff}[3][{}]{{\dfrac{\ud^{#1} {#2}}{\ud {#3}{}^{#1}}}}
\newcommand{\psddiff}[3][{}]{{\frac{\partial^{#1}}{\partial{}^{#1} {#3}} {#2}}}
\newcommand{\sddiff}[3][{}]{{\frac{\ud^{#1}}{\ud {#3}{}^{#1}} {#2}}}

\begin{document}
\tikzstyle{every picture}+=[remember picture]
\title{Confinement induced resonance}
\author{Yichao Yu}
\email{yuyichao@mit.edu}
\homepage{http://yyc-arch.org/}
\date{\today}
\affiliation{Harvard Department of Physics}

\begin{abstract}
\end{abstract}

\maketitle
%%%%%%%%%%%%%%%%%%%%%%%%%%%%%%%%%%%%%%%%%%%%%%%%%%%%%%%%%%%%%%%%%%
\section{Introduction}

\section{Feshbach Resonance}
\subsection{$T$-Matrix and scatter length}

% S Wave (L=0 isotropic)
% k \rightarrow 0
% r \rightarrow \infty
\eqar{
  \Psi\paren{\vec r}=&\ue^{\ui \vec k_0\cdot \vec r}-\Psi_{s}
  \approx&\ue^{\ui \vec k_0\cdot \vec r}-\frac{a}{r}
}

% Schroedinger equation
% E=\frac{k^2}{m}
% U(\vec k, \vec k')=\int\ud^3\vec r\int\ud^3\vec r'U\paren{\vec r-\vec r'}\ue^{\ui\vec k\cdot\vec r}\ue^{-\ui\vec k'\cdot\vec r'}
\eqar{
  \frac1{m}\paren{k^2-k'^2}\Psi_{s}\paren{\vec k'}=&U\paren{\vec k', \vec k}+\int\frac{\ud^3 k''}{\paren{2\pi}^3}U\paren{\vec k', \vec k''}\Psi_s\paren{\vec k''}
}

Define $T$ matrix
\eqar{
  T\paren{\vec k', \vec k}=&U\paren{\vec k', \vec k}+\int\frac{\ud^3k''}{\paren{2\pi}^3}\frac{U\paren{\vec k', \vec k''}}{k^2/m-k''^2/m+\ui 0}T\paren{\vec k'', \vec k}\\
  \Psi_s\paren{\vec k'}=&\frac{m}{k^2-k'^2+\ui 0}T\paren{\vec k', \vec k, \frac{k^2}{m}}\\
  f\paren{\vec k', \vec k}=&-\frac{m}{4\pi\hbar^2}T\paren{\vec k', \vec k, \frac{k^2}{m}}
}

% Effective interaction
% Energy cut-off \varepsilon_c
% k_c, R
\eqar{
  \tilde U\paren{\vec k', \vec k}=&U\paren{\vec k', \vec k}+\int_{k''^2/m>\varepsilon_c}\frac{\ud^3 k''}{\paren{2\pi}^3}\frac{\tilde U\paren{\vec k', \vec k''}}{k^2/m-k''^2/m+\ui 0}\tilde U\paren{\vec k'', \vec k}
}
\eqar{
  T\paren{\vec k', \vec k}=&\tilde U\paren{\vec k', \vec k}+\int_{k''^2/m<\varepsilon_c}\frac{\ud^3 k'}{\paren{2\pi}^3}\frac{\tilde U\paren{\vec k', \vec k''}}{k^2/m-k''^2/m+\ui0}T\paren{\vec k'', \vec k}
}
% U\paren{\vec r}=U_0\delta^3\paren{\vec r}
% Energy conservation
% mainly k' == k
\eqar{
  \frac{1}{f\paren{k}}=&-\frac{4\pi}{mU_0}+4\pi\int_{\abs{q}<1/R}\frac{\ud^3q}{\paren{2\pi}^3}\frac{1}{k^2-q^2+\ui0}\\
  =&\frac{1}{2\pi^2}\paren{-\frac{1}{R}-\frac{k}{2}\ln\paren{-\frac{R^{-1}-k-\ui0}{R^{-1}+k+\ui0}}}\\
  \approx&\frac{1}{2\pi^2}\paren{-\frac{1}{R}-\frac{\ui\pi k}{2}+k^2R}
}

% f\paren{0}=-a
\eqar{
  a=&\frac{\pi}{2}\frac{R}{1+\frac{2\pi^2R}{mU_0}}\\
  f\paren{k}=&\frac{1}{a^{-1}+r_{eff}k^2/2-\ui k}
}
% Dependency of U_0 on R

\subsection{Multichannel scattering and Feshbach resonance}
% Hamiltonian
\eqar{
  \hat H_0|\vec k\rangle=&
}
% Solution

% Bound states

% Compare to generic fomula and get scattering length

\section{Confinement induced resonance}
% Seperate COM and relative

% Cutoff

% Compare to 3-d

% Bound state

% Scattering length

\section{Conclusion}

\bibliography{paper}
\end{document}
