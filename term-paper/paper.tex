\documentclass[aps,twocolumn,secnumarabic,balancelastpage,amsmath,amssymb,nofootinbib]{revtex4}
\usepackage{amsmath}
\usepackage{amssymb}
\usepackage{amsfonts}
\usepackage{chapterbib}
\usepackage{color}
\usepackage{graphics}
\usepackage[pdftex]{graphicx}
\usepackage{grffile}
\usepackage{longtable}
\usepackage{epsf}
\usepackage{bm}
\usepackage{tikz}
\usepackage{asymptote}
\usepackage{thumbpdf}
\usepackage{bbm}
\usepackage{amscd}
\usepackage{units}
\usepackage{textcomp}
\usepackage[utf8x]{inputenc}
\usepackage{feyn}
\usepackage{feynmp}
\usepackage[colorlinks=true]{hyperref}
\newcommand{\drelectron}[1]{\node at #1 [circle, draw, inner sep=0pt, minimum size=1pt] {\_}}

\newcommand{\ud}{\mathrm{d}}
\newcommand{\ue}{\mathrm{e}}
\newcommand{\ui}{\mathrm{i}}
\newcommand{\res}{\mathrm{Res}}
\newcommand{\Tr}{\mathrm{Tr}}
\newcommand{\dsum}{\displaystyle\sum}
\newcommand{\dprod}{\displaystyle\prod}
\newcommand{\dlim}{\displaystyle\lim}
\newcommand{\dint}{\displaystyle\int}
\newcommand{\fsno}[1]{{\!\not\!{#1}}}
\newcommand{\eqar}[1]
{
  \begin{align*}
    #1
  \end{align*}
}
\newcommand{\eqarn}[1]
{
  \begin{align}
    #1
  \end{align}
}
\newcommand{\texp}[2]{\ensuremath{{#1}\times10^{#2}}}
\newcommand{\dexp}[2]{\ensuremath{{#1}\cdot10^{#2}}}
\newcommand{\eval}[2]{{\left.{#1}\right|_{#2}}}
\newcommand{\paren}[1]{{\left({#1}\right)}}
\newcommand{\lparen}[1]{{\left({#1}\right.}}
\newcommand{\rparen}[1]{{\left.{#1}\right)}}
\newcommand{\abs}[1]{{\left|{#1}\right|}}
\newcommand{\sqr}[1]{{\left[{#1}\right]}}
\newcommand{\crly}[1]{{\left\{{#1}\right\}}}
\newcommand{\angl}[1]{{\left\langle{#1}\right\rangle}}
\newcommand{\tpdiff}[4][{}]{{\paren{\frac{\partial^{#1} {#2}}{\partial {#3}{}^{#1}}}_{#4}}}
\newcommand{\tpsdiff}[4][{}]{{\paren{\frac{\partial^{#1}}{\partial {#3}{}^{#1}}{#2}}_{#4}}}
\newcommand{\pdiff}[3][{}]{{\frac{\partial^{#1} {#2}}{\partial {#3}{}^{#1}}}}
\newcommand{\diff}[3][{}]{{\frac{\ud^{#1} {#2}}{\ud {#3}{}^{#1}}}}
\newcommand{\psdiff}[3][{}]{{\frac{\partial^{#1}}{\partial {#3}{}^{#1}} {#2}}}
\newcommand{\sdiff}[3][{}]{{\frac{\ud^{#1}}{\ud {#3}{}^{#1}} {#2}}}
\newcommand{\tpddiff}[4][{}]{{\left(\dfrac{\partial^{#1} {#2}}{\partial {#3}{}^{#1}}\right)_{#4}}}
\newcommand{\tpsddiff}[4][{}]{{\paren{\dfrac{\partial^{#1}}{\partial {#3}{}^{#1}}{#2}}_{#4}}}
\newcommand{\pddiff}[3][{}]{{\dfrac{\partial^{#1} {#2}}{\partial {#3}{}^{#1}}}}
\newcommand{\ddiff}[3][{}]{{\dfrac{\ud^{#1} {#2}}{\ud {#3}{}^{#1}}}}
\newcommand{\psddiff}[3][{}]{{\frac{\partial^{#1}}{\partial{}^{#1} {#3}} {#2}}}
\newcommand{\sddiff}[3][{}]{{\frac{\ud^{#1}}{\ud {#3}{}^{#1}} {#2}}}

\begin{document}
\tikzstyle{every picture}+=[remember picture]
\title{Confinement induced resonance}
\author{Yichao Yu}
\email{yuyichao@mit.edu}
\homepage{http://yyc-arch.org/}
\date{\today}
\affiliation{Harvard Department of Physics}

\begin{abstract}
  We present a formal derivation of the confinement induced resonance for particles in a lower dimension. The generic $T$-matrix method is used to calculate the $S$-wave collisional property between two ultra-code atoms. We show the breakdown of a naive approach to describe such low dimensional system near a Feshbach resonance and preform a real three dimensional calculation to include additional effect from the confinement. The normalization of the high energy states in such cases are done by comparing to the result in free space.
\end{abstract}

\maketitle
%%%%%%%%%%%%%%%%%%%%%%%%%%%%%%%%%%%%%%%%%%%%%%%%%%%%%%%%%%%%%%%%%%
\section{Introduction}
In the field of ultra-cold atoms and molecules, the collisional properties between the particles play an important role for cooling into quantum degeneracy. The tunability of the scattering length using Feshbach resonance opens the door for engineering the interaction and using it as a source of correlation and entanglement in a many-body system.


\section{Feshbach Resonance}
% Cite optical feshbach resonance
Feshbach resonance is a way to tune the interaction between two atoms. It appears as a divergence in the scattering length when the energy of a molecular bound state matches or is closed to the energy of the free atom state. The relative energy between the molecular and atomic is usually tuned using the magnetic field and the difference in Zeeman effect between the two states although optical dressing has also been used on excited metastable state.\\

In this section, we will describe a formal way to calculate the properties of $S$-wave scattering at low temperature. We will then apply this framework to a simplified model of Feshbach resonance consists of a open channel for the free atomic state and a closed channel for the molecular bound state.

\subsection{Wave function of a scattering process}
In a scattering problem, the wave function can be broken into the incident wave $\Psi_i$ and the scattered wave $\Psi_s$ where the incident wave is usually a plain wave
\[\Psi_i=\ue^{\ui\vec k\cdot\vec r}\]
and the scattered wave can be expressed as a spherical wave centered around the scattering core ($\vec r=0$)
\[\Psi_s=\frac{f\paren{\theta}}{r}\ue^{\ui k'r}\]
For all the discussion in this paper, we are taking the limit of low temperature (i.e. low kinetic energy) in which case the scattering is limitted to $S$-wave (angular momentum $L^2=0$). We also assume that the interaction is short range so that the behavior is regular for long wavelength and we can take the limit of $k\rightarrow0$. In such a limit, the total wave function can be written as
\eqar{
  \Psi\paren{\vec r}=&\Psi_i\paren{\vec r}+\Psi_s\paren{\vec r}\\
  \approx&\ue^{\ui \vec k_0\cdot \vec r}-\frac{a}{r}
}
where $a$, which describe the fraction being scattered, is defined as the scattering length. In order to solve this problem, it is convinient to express the Schr\"odinger equation in the momentum space
\eqar{
  \frac{k^2-k'^2}{m}\Psi_{s}\paren{\vec k'}=&U\paren{\vec k', \vec k}+\int\frac{\ud^3 k''}{\paren{2\pi}^3}U\paren{\vec k', \vec k''}\Psi_s\paren{\vec k''}
}
where $U\paren{\vec k', \vec k}$ is the Fourier transform of the interaction
\[ U(\vec k', \vec k)=\int\ud^3\vec r\int\ud^3\vec r'U\paren{\vec r'-\vec r}\ue^{\ui\vec k'\cdot\vec r'}\ue^{-\ui\vec k\cdot\vec r}\]
and we have used the fact that the interaction is short range and the energy of the state is determined by the long distance behavior of the incident state $E_k=k^2/2m$.

\subsection{$T$-Matrix}
The $T$-matrix is defined as a map from the initial state of the scattering (which is a $\delta$ function in $k$ space in our case) to the final state. It is defined as
\eqar{
  T\paren{\vec k', \vec k}=&U\paren{\vec k', \vec k}+m\int\frac{\ud^3k''}{\paren{2\pi}^3}\frac{U\paren{\vec k', \vec k''}}{k^2-k''^2+\ui 0}T\paren{\vec k'', \vec k}}
where the $+\ui 0$ denotes a infinitesimal quantity to ensure that we have outgoing waves. Note the similarity with the equation satisfied by the scattered wavefunction. Using the $T$-matrix, we can express the scattered wavefunction as
\[\Psi_s\paren{\vec k'}=\frac{m}{k^2-k'^2+\ui 0}T\paren{\vec k', \vec k}\]
And we cans also define the scattering amplitude as
% \[f\paren{\vec k', \vec k}=-\frac{m}{4\pi\hbar^2}T\paren{\vec k', \vec k}\]
\[f\paren{\vec k', \vec k}=-\frac{m}{4\pi}T\paren{\vec k', \vec k}\]
For the problem we are considering, we will take the limit of $k, k'\rightarrow0$, therefore
\[\Psi_s\paren{\vec k'}=\frac{m}{k^2-k'^2+\ui 0}T\paren{0, 0}\]
In real space
\eqar{
  \Psi_s\paren{\vec r}=&\int\frac{\ud k'^3}{\paren{2\pi}^3}\frac{m\ue^{\ui\vec k'\cdot\vec r}}{k^2-k'^2+\ui 0}T\paren{0, 0}\\
  =&-\frac{m\ue^{\ui kr}}{4\pi r}T\paren{0, 0}
}
and by comparing to the form of $\Psi_s$ at low temperature we have $f(k\rightarrow0)=-a$

\subsection{Effective interaction}
Althought the interaction potential between two atoms can be very strong and hard to calculate, they are usually very short range and the precise behavior can only affect the states with a large $k$. Therefore, for the state

% Energy cut-off \varepsilon_c
% k_c, R
\eqar{
  \tilde U\paren{\vec k', \vec k}=&U\paren{\vec k', \vec k}+\int_{k''^2/m>\varepsilon_c}\frac{\ud^3 k''}{\paren{2\pi}^3}\frac{\tilde U\paren{\vec k', \vec k''}}{k^2/m-k''^2/m+\ui 0}\tilde U\paren{\vec k'', \vec k}
}
\eqar{
  T\paren{\vec k', \vec k}=&\tilde U\paren{\vec k', \vec k}+\int_{k''^2/m<\varepsilon_c}\frac{\ud^3 k'}{\paren{2\pi}^3}\frac{\tilde U\paren{\vec k', \vec k''}}{k^2/m-k''^2/m+\ui0}T\paren{\vec k'', \vec k}
}
% U\paren{\vec r}=U_0\delta^3\paren{\vec r}
% Energy conservation
% mainly k' == k
\eqar{
  \frac{1}{f\paren{k}}=&-\frac{4\pi}{mU_0}+4\pi\int_{\abs{q}<1/R}\frac{\ud^3q}{\paren{2\pi}^3}\frac{1}{k^2-q^2+\ui0}\\
  =&\frac{1}{2\pi^2}\paren{-\frac{1}{R}-\frac{k}{2}\ln\paren{-\frac{R^{-1}-k-\ui0}{R^{-1}+k+\ui0}}}\\
  \approx&\frac{1}{2\pi^2}\paren{-\frac{1}{R}-\frac{\ui\pi k}{2}+k^2R}
}

% f\paren{0}=-a
\eqar{
  a=&\frac{\pi}{2}\frac{R}{1+\frac{2\pi^2R}{mU_0}}\\
  f\paren{k}=&\frac{1}{a^{-1}+r_{eff}k^2/2-\ui k}
}
% Dependency of U_0 on R

\subsection{Multichannel scattering and Feshbach resonance}
% Hamiltonian
\eqar{
  \hat H_0|\vec k\rangle=&
}
% Solution

% Bound states

% Compare to generic fomula and get scattering length

\section{Confinement induced resonance}
% Seperate COM and relative

% Cutoff

% Compare to 3-d

% Bound state

% Scattering length

\section{Conclusion}

\bibliography{paper}
\end{document}
