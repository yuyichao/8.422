\documentclass[10pt,fleqn]{article}
\newcommand{\name}[1]{\def\psettitlename{#1}}
\newcommand{\course}[1]{\def\psettitlecourse{#1}}
\newcommand{\rsection}[1]{\def\psettitlersection{#1}}
\newcommand{\psetnum}[1]{\def\psettitlepsetnum{#1}}
% \usepackage[journal=rsc]{chemstyle}
% \usepackage{mhchem}
\usepackage{amsmath}
\usepackage{amssymb}
\usepackage{amsfonts}
\usepackage{esint}
\usepackage{bbm}
\usepackage{amscd}
\usepackage{picinpar}
\usepackage{graphicx}
\usepackage{tikz}
\usepackage{tikz-3dplot}
\usepackage{indentfirst}
\usepackage{wrapfig}
\usepackage{units}
\usepackage{textcomp}
\usepackage[utf8x]{inputenc}
% \usepackage{feyn}
\usepackage{feynmp}
\usepackage{xkeyval}
\usepackage{xargs}
\usepackage{verbatim}
\usepackage{pgfplots}
\usetikzlibrary{
  arrows,
  calc,
  decorations.pathmorphing,
  decorations.pathreplacing,
  decorations.markings,
  fadings,
  positioning,
  shapes
}

\DeclareGraphicsRule{*}{mps}{*}{}
\newcommand{\ud}{\mathrm{d}}
\newcommand{\ue}{\mathrm{e}}
\newcommand{\ui}{\mathrm{i}}
\newcommand{\res}{\mathrm{Res}}
\newcommand{\Tr}{\mathrm{Tr}}
\newcommand{\dsum}{\displaystyle\sum}
\newcommand{\dprod}{\displaystyle\prod}
\newcommand{\dlim}{\displaystyle\lim}
\newcommand{\dint}{\displaystyle\int}
\newcommand{\fsno}[1]{{\!\not\!{#1}}}
\newcommand{\eqar}[1]
{
  \begin{align*}
    #1
  \end{align*}
}
\newcommand{\texp}[2]{\ensuremath{{#1}\times10^{#2}}}
\newcommand{\dexp}[2]{\ensuremath{{#1}\cdot10^{#2}}}
\newcommand{\eval}[2]{{\left.{#1}\right|_{#2}}}
\newcommand{\paren}[1]{{\left({#1}\right)}}
\newcommand{\lparen}[1]{{\left({#1}\right.}}
\newcommand{\rparen}[1]{{\left.{#1}\right)}}
\newcommand{\abs}[1]{{\left|{#1}\right|}}
\newcommand{\sqr}[1]{{\left[{#1}\right]}}
\newcommand{\crly}[1]{{\left\{{#1}\right\}}}
\newcommand{\angl}[1]{{\left\langle{#1}\right\rangle}}
\newcommand{\tpdiff}[4][{}]{{\paren{\frac{\partial^{#1} {#2}}{\partial {#3}{}^{#1}}}_{#4}}}
\newcommand{\tpsdiff}[4][{}]{{\paren{\frac{\partial^{#1}}{\partial {#3}{}^{#1}}{#2}}_{#4}}}
\newcommand{\pdiff}[3][{}]{{\frac{\partial^{#1} {#2}}{\partial {#3}{}^{#1}}}}
\newcommand{\diff}[3][{}]{{\frac{\ud^{#1} {#2}}{\ud {#3}{}^{#1}}}}
\newcommand{\psdiff}[3][{}]{{\frac{\partial^{#1}}{\partial {#3}{}^{#1}} {#2}}}
\newcommand{\sdiff}[3][{}]{{\frac{\ud^{#1}}{\ud {#3}{}^{#1}} {#2}}}
\newcommand{\tpddiff}[4][{}]{{\left(\dfrac{\partial^{#1} {#2}}{\partial {#3}{}^{#1}}\right)_{#4}}}
\newcommand{\tpsddiff}[4][{}]{{\paren{\dfrac{\partial^{#1}}{\partial {#3}{}^{#1}}{#2}}_{#4}}}
\newcommand{\pddiff}[3][{}]{{\dfrac{\partial^{#1} {#2}}{\partial {#3}{}^{#1}}}}
\newcommand{\ddiff}[3][{}]{{\dfrac{\ud^{#1} {#2}}{\ud {#3}{}^{#1}}}}
\newcommand{\psddiff}[3][{}]{{\frac{\partial^{#1}}{\partial{}^{#1} {#3}} {#2}}}
\newcommand{\sddiff}[3][{}]{{\frac{\ud^{#1}}{\ud {#3}{}^{#1}} {#2}}}
\usepackage{fancyhdr}
\usepackage{multirow}
\usepackage{fontenc}
% \usepackage{tipa}
\usepackage{ulem}
\usepackage{color}
\usepackage{cancel}
\newcommand{\hcancel}[2][black]{\setbox0=\hbox{#2}%
  \rlap{\raisebox{.45\ht0}{\textcolor{#1}{\rule{\wd0}{1pt}}}}#2}
\pagestyle{fancy}
\setlength{\headheight}{67pt}
\fancyhead{}
\fancyfoot{}
\fancyfoot[C]{\thepage}
\fancyhead[R]
{
  \psettitlename \\
  \psettitlecourse{} Problem Set \psettitlepsetnum \\
  \ifx\psettitlersection\empty
  \else
  Recitation Section \psettitlersection
  \fi
}
\renewcommand{\footruleskip}{0pt}
\renewcommand{\headrulewidth}{0.4pt}
\renewcommand{\footrulewidth}{0pt}

\newcommand\pgfmathsinandcos[3]{%
  \pgfmathsetmacro#1{sin(#3)}%
  \pgfmathsetmacro#2{cos(#3)}%
}
\newcommand\LongitudePlane[3][current plane]{%
  \pgfmathsinandcos\sinEl\cosEl{#2} % elevation
  \pgfmathsinandcos\sint\cost{#3} % azimuth
  \tikzset{#1/.estyle={cm={\cost,\sint*\sinEl,0,\cosEl,(0,0)}}}
}
\newcommand\LatitudePlane[3][current plane]{%
  \pgfmathsinandcos\sinEl\cosEl{#2} % elevation
  \pgfmathsinandcos\sint\cost{#3} % latitude
  \pgfmathsetmacro\yshift{\cosEl*\sint}
  \tikzset{#1/.estyle={cm={\cost,0,0,\cost*\sinEl,(0,\yshift)}}} %
}
\newcommand\DrawLongitudeCircle[2][1]{
  \LongitudePlane{\angEl}{#2}
  \tikzset{current plane/.prefix style={scale=#1}}
  % angle of "visibility"
  \pgfmathsetmacro\angVis{atan(sin(#2)*cos(\angEl)/sin(\angEl))} %
  \draw[current plane] (\angVis:1) arc (\angVis:\angVis+180:1);
  \draw[current plane,dashed] (\angVis-180:1) arc (\angVis-180:\angVis:1);
}
\newcommand\DrawLatitudeCircleArrow[2][1]{
  \LatitudePlane{\angEl}{#2}
  \tikzset{current plane/.prefix style={scale=#1}}
  \pgfmathsetmacro\sinVis{sin(#2)/cos(#2)*sin(\angEl)/cos(\angEl)}
  % angle of "visibility"
  \pgfmathsetmacro\angVis{asin(min(1,max(\sinVis,-1)))}
  \draw[current plane,decoration={markings, mark=at position 0.6 with {\arrow{<}}},postaction={decorate},line width=.6mm] (\angVis:1) arc (\angVis:-\angVis-180:1);
  \draw[current plane,dashed,line width=.6mm] (180-\angVis:1) arc (180-\angVis:\angVis:1);
}
\newcommand\DrawLatitudeCircle[2][1]{
  \LatitudePlane{\angEl}{#2}
  \tikzset{current plane/.prefix style={scale=#1}}
  \pgfmathsetmacro\sinVis{sin(#2)/cos(#2)*sin(\angEl)/cos(\angEl)}
  % angle of "visibility"
  \pgfmathsetmacro\angVis{asin(min(1,max(\sinVis,-1)))}
  \draw[current plane] (\angVis:1) arc (\angVis:-\angVis-180:1);
  \draw[current plane,dashed] (180-\angVis:1) arc (180-\angVis:\angVis:1);
}
\newcommand\coil[1]{
  {\rh * cos(\t * pi r)}, {\apart * (2 * #1 + \t) + \rv * sin(\t * pi r)}
}
\makeatletter
\define@key{DrawFromCenter}{style}[{->}]{
  \tikzset{DrawFromCenterPlane/.style={#1}}
}
\define@key{DrawFromCenter}{r}[1]{
  \def\@R{#1}
}
\define@key{DrawFromCenter}{center}[(0, 0)]{
  \def\@Center{#1}
}
\define@key{DrawFromCenter}{theta}[0]{
  \def\@Theta{#1}
}
\define@key{DrawFromCenter}{phi}[0]{
  \def\@Phi{#1}
}
\presetkeys{DrawFromCenter}{style, r, center, theta, phi}{}
\newcommand*\DrawFromCenter[1][]{
  \setkeys{DrawFromCenter}{#1}{
    \pgfmathsinandcos\sint\cost{\@Theta}
    \pgfmathsinandcos\sinp\cosp{\@Phi}
    \pgfmathsinandcos\sinA\cosA{\angEl}
    \pgfmathsetmacro\DX{\@R*\cost*\cosp}
    \pgfmathsetmacro\DY{\@R*(\cost*\sinp*\sinA+\sint*\cosA)}
    \draw[DrawFromCenterPlane] \@Center -- ++(\DX, \DY);
  }
}
\newcommand*\DrawFromCenterText[2][]{
  \setkeys{DrawFromCenter}{#1}{
    \pgfmathsinandcos\sint\cost{\@Theta}
    \pgfmathsinandcos\sinp\cosp{\@Phi}
    \pgfmathsinandcos\sinA\cosA{\angEl}
    \pgfmathsetmacro\DX{\@R*\cost*\cosp}
    \pgfmathsetmacro\DY{\@R*(\cost*\sinp*\sinA+\sint*\cosA)}
    \draw[DrawFromCenterPlane] \@Center -- ++(\DX, \DY) node {#2};
  }
}
\makeatother
\tikzstyle{snakearrow} = [decorate, decoration={pre length=0.2cm,
  post length=0.2cm, snake, amplitude=.4mm,
  segment length=2mm},thick, ->]
%% document-wide tikz options and styles
\tikzset{%
  >=latex, % option for nice arrows
  inner sep=0pt,%
  outer sep=2pt,%
  mark coordinate/.style={inner sep=0pt,outer sep=0pt,minimum size=3pt,
    fill=black,circle}%
}
\addtolength{\hoffset}{-1.3cm}
\addtolength{\voffset}{-2cm}
\addtolength{\textwidth}{3cm}
\addtolength{\textheight}{2.5cm}
\renewcommand{\footskip}{10pt}
\setlength{\headwidth}{\textwidth}
\setlength{\headsep}{20pt}
\setlength{\marginparwidth}{0pt}
\parindent=0pt
\psetnum{10}
\course{8.422}
\rsection{1}
\name{Yichao Yu}
\renewcommand{\thesection}{\arabic{section}.}
\renewcommand{\thesubsection}{(\alph{subsection})}
\renewcommand{\thesubsubsection}{\roman{subsubsection}.}

\begin{document}
\section{The Dressed Atom}
\subsection{}
\subsection{}
\subsection{}
\subsection{}
\subsection{}
\subsection{}

\section{Sideband Cooling}
\subsection{}
Interaction term
\eqar{
  H_I=&\hbar\Omega\paren{\sigma_++\sigma_-}\cos\paren{k\hat x-\omega t}\\
  =&\hbar\Omega\paren{\sigma_++\sigma_-}\cos\paren{\eta\paren{a+a^\dagger}-\omega t}\\
  =&\frac{\hbar\Omega}2\paren{\sigma_++\sigma_-}\paren{\ue^{\ui\paren{\eta\paren{a+a^\dagger}-\omega t}}+\ue^{-\ui\paren{\eta\paren{a+a^\dagger}-\omega t}}}
  \intertext{In the Lamb-Dicke limit}
  H_I=&\frac{\hbar\Omega}2\paren{\sigma_++\sigma_-}\paren{\ue^{\ui\eta\paren{a+a^\dagger}}\ue^{-\ui\omega t}+\ue^{-\ui\eta\paren{a+a^\dagger}}\ue^{\ui\omega t}}\\
  =&\frac{\hbar\Omega}2\paren{\sigma_++\sigma_-}\paren{\paren{1+\ui\eta\paren{a+a^\dagger}}\ue^{-\ui\omega t}+\paren{1-\ui\eta\paren{a+a^\dagger}}\ue^{\ui\omega t}}\\
  =&\frac{\hbar\Omega}2\paren{\sigma_++\sigma_-}\paren{\ue^{-\ui\omega t}+\ue^{\ui\omega t}}+\ui\eta\frac{\hbar\Omega}2\paren{\sigma_++\sigma_-}\paren{a+a^\dagger}\paren{\ue^{-\ui\omega t}-\ue^{\ui\omega t}}
  \intertext{In the interaction picture}
  \sigma_\pm'=&\ue^{\ui H_0t/\hbar}\sigma_\pm\ue^{-\ui H_0t/\hbar}\\
  =&\sigma_\pm\ue^{\ui\omega_0 t}\\
  a'=&\ue^{\ui H_0t/\hbar}a\ue^{-\ui H_0t/\hbar}\\
  =&a\ue^{-\ui\nu t}\\
  a'^\dagger=&\ue^{\ui H_0t/\hbar}a^\dagger\ue^{-\ui H_0t/\hbar}\\
  =&a^\dagger\ue^{\ui\nu t}
  \intertext{In the rotating wave approximation}
  H_I'=&\ue^{\ui H_0t/\hbar}H_I\ue^{-\ui H_0t/\hbar}\\
  =&\frac{\hbar\Omega}2\paren{\sigma_+\ue^{-\ui\delta t}+\sigma_-\ue^{\ui\delta t}}
  +\ui\eta\frac{\hbar\Omega}2\paren{\sigma_+a\ue^{-\ui\paren{\delta+\nu}t}-\sigma_-a^\dagger\ue^{\ui\paren{\delta+\nu}t}}\\
  &+\ui\eta\frac{\hbar\Omega}2\paren{\sigma_+a^\dagger\ue^{-\ui\paren{\delta-\nu}t}-\sigma_-a\ue^{\ui\paren{\delta-\nu}t}}
}
When the light is in resonance with the sideband, the Rabi frequencies
\eqar{
  \Omega_{n,n-1}=&\abs{\langle n-1,e|\ui\eta\frac{\hbar\Omega}2\paren{\sigma_+a-\sigma_-a^\dagger}|n,g\rangle}\\
  =&\frac{\eta\hbar\Omega}2\langle n-1|a|n\rangle\\
  =&\frac{\eta\hbar\Omega}2\sqrt{n}\\
  \Omega_{n,n+1}=&\abs{\langle n+1,e|\ui\eta\frac{\hbar\Omega}2\paren{\sigma_+a^\dagger-\sigma_-a}|n,g\rangle}\\
  =&\frac{\eta\hbar\Omega}2\langle n+1|a^\dagger|n\rangle\\
  =&\frac{\eta\hbar\Omega}2\sqrt{n+1}
}
\subsection{}
For spontanious decay, when the emitted photon is $\theta$ from the axis of the trap, the probability to go from $n$ to $n+1$ is $\eta^2\cos^2\theta\paren{n+1}$ Assuming isotropic emission patter (i.e. non-polarized) the averate probability is
\eqar{
  p_{n,n+1}=&\frac1{4\pi}\int_0^{\pi}\ud\theta\int_{0}^{2\pi}\sin\theta\ud\phi\eta^2\cos^2\theta\paren{n+1}\\
  =&\frac{\eta^2\paren{n+1}}{2}\int_{-1}^{1}\ud z z^2\\
  =&\frac{\eta^2\paren{n+1}}{3}
  \intertext{Similarly, the probability of going to $n-1$ is}
  p_{n,n-1}=&\frac{\eta^2n}{3}
}
To the lowest order in $\eta$, the rate at which $|n,g\rangle\rightarrow|n,e\rangle\rightarrow|n-1,g\rangle$ happens is
\eqar{
  &\frac{n\Omega^2\eta^2}{3}\frac{\Gamma}{\Gamma^2+4\delta^2}
  \intertext{For $|n,g\rangle\rightarrow|n-1,e\rangle\rightarrow|n-1,g\rangle$}
  &n\Omega^2\eta^2\frac{\Gamma}{\Gamma^2+4\paren{\delta+\nu}^2}\\
  A_-=&\Omega^2\Gamma\eta^2\paren{\frac{1}{\Gamma^2+4\paren{\delta+\nu}^2}+\frac{1}{3}\frac{1}{\Gamma^2+4\delta^2}}
  \intertext{Similarly}
  A_+=&\Omega^2\Gamma\eta^2\paren{\frac{1}{\Gamma^2+4\paren{\delta-\nu}^2}+\frac{1}{3}\frac{1}{\Gamma^2+4\delta^2}}
}
\subsection{}
\eqar{
  \diff{\angl{n}}{t}=&\sum_{n=0}^{\infty}n\diff{p_n}{t}\\
  =&\sum_{n=0}^{\infty}n\paren{nA_+p_{n-1}+\paren{n+1}A_-p_{n+1}-\paren{n+1}A_+p_n-nA_-p_n}\\
  =&\sum_{n=0}^{\infty}\paren{n^2A_+p_{n-1}+n\paren{n+1}A_-p_{n+1}-n\paren{n+1}A_+p_n-n^2A_-p_n}\\
  =&\sum_{n=0}^{\infty}\paren{\paren{n+1}^2A_+p_n+\paren{n-1}nA_-p_n-n\paren{n+1}A_+p_n-n^2A_-p_n}\\
  =&\sum_{n=0}^{\infty}\paren{\paren{n+1}A_+p_n-nA_-p_n}\\
  =&A_+-\paren{A_--A_+}\angl{n}
}
The solution is a exponential decay to $\angl{n}=\dfrac{A_+}{A_--A_+}$ with decay rate $A_--A_+$. For driving cooling sideband in the resolved limit
\eqar{
  A_-\approx&\frac{\Omega^2\eta^2}{\Gamma}\\
  A_+\approx&\frac{7\Omega^2\Gamma\eta^2}{48\nu^2}
}
Final temperature
\eqar{
  \angl{n}_\infty\approx&\dfrac{A_+}{A_-}\\
  =&\frac{7\Gamma^2}{48\nu^2}\\
  T_\infty=&\angl{n}_\infty\frac{\hbar\nu}{k_B}\\
  \approx&\frac{7\hbar\Gamma^2}{48k_B\nu}
}
Decay time
\eqar{
  \tau\approx&\frac{1}{A_-}\\
  \approx&\frac{\Gamma}{\Omega^2\eta^2}
}
\subsection{}
For the narrow line
\eqar{
  \eta=&k\sqrt{\frac{\hbar}{2m\nu}}\\
  =&0.022\\
  T_\infty\approx&\frac{7\hbar\Gamma^2}{48k_B\nu}\\
  =&0.11\text{aK}\\
  \tau\approx&\frac{\Gamma}{\Omega^2\eta^2}\\
  =&22\text{hr}
}
For broadened line
\eqar{
  T_\infty\approx&\frac{7\hbar\Gamma^2}{48k_B\nu}\\
  =&7.0\text{nK}\\
  \tau\approx&\frac{\Gamma}{\Omega^2\eta^2}\\
  =&0.32\text{s}
}

\section{Optical dipole trap}
\subsection{}
\subsection{}
\subsection{}

\end{document}
