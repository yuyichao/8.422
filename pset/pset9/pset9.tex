\documentclass[10pt,fleqn]{article}
\newcommand{\name}[1]{\def\psettitlename{#1}}
\newcommand{\course}[1]{\def\psettitlecourse{#1}}
\newcommand{\rsection}[1]{\def\psettitlersection{#1}}
\newcommand{\psetnum}[1]{\def\psettitlepsetnum{#1}}
% \usepackage[journal=rsc]{chemstyle}
% \usepackage{mhchem}
\usepackage{amsmath}
\usepackage{amssymb}
\usepackage{amsfonts}
\usepackage{esint}
\usepackage{bbm}
\usepackage{amscd}
\usepackage{picinpar}
\usepackage{graphicx}
\usepackage{tikz}
\usepackage{tikz-3dplot}
\usepackage{indentfirst}
\usepackage{wrapfig}
\usepackage{units}
\usepackage{textcomp}
\usepackage[utf8x]{inputenc}
% \usepackage{feyn}
\usepackage{feynmp}
\usepackage{xkeyval}
\usepackage{xargs}
\usepackage{verbatim}
\usepackage{pgfplots}
\usetikzlibrary{
  arrows,
  calc,
  decorations.pathmorphing,
  decorations.pathreplacing,
  decorations.markings,
  fadings,
  positioning,
  shapes
}

\DeclareGraphicsRule{*}{mps}{*}{}
\newcommand{\ud}{\mathrm{d}}
\newcommand{\ue}{\mathrm{e}}
\newcommand{\ui}{\mathrm{i}}
\newcommand{\res}{\mathrm{Res}}
\newcommand{\Tr}{\mathrm{Tr}}
\newcommand{\dsum}{\displaystyle\sum}
\newcommand{\dprod}{\displaystyle\prod}
\newcommand{\dlim}{\displaystyle\lim}
\newcommand{\dint}{\displaystyle\int}
\newcommand{\fsno}[1]{{\!\not\!{#1}}}
\newcommand{\eqar}[1]
{
  \begin{align*}
    #1
  \end{align*}
}
\newcommand{\texp}[2]{\ensuremath{{#1}\times10^{#2}}}
\newcommand{\dexp}[2]{\ensuremath{{#1}\cdot10^{#2}}}
\newcommand{\eval}[2]{{\left.{#1}\right|_{#2}}}
\newcommand{\paren}[1]{{\left({#1}\right)}}
\newcommand{\lparen}[1]{{\left({#1}\right.}}
\newcommand{\rparen}[1]{{\left.{#1}\right)}}
\newcommand{\abs}[1]{{\left|{#1}\right|}}
\newcommand{\sqr}[1]{{\left[{#1}\right]}}
\newcommand{\crly}[1]{{\left\{{#1}\right\}}}
\newcommand{\angl}[1]{{\left\langle{#1}\right\rangle}}
\newcommand{\tpdiff}[4][{}]{{\paren{\frac{\partial^{#1} {#2}}{\partial {#3}{}^{#1}}}_{#4}}}
\newcommand{\tpsdiff}[4][{}]{{\paren{\frac{\partial^{#1}}{\partial {#3}{}^{#1}}{#2}}_{#4}}}
\newcommand{\pdiff}[3][{}]{{\frac{\partial^{#1} {#2}}{\partial {#3}{}^{#1}}}}
\newcommand{\diff}[3][{}]{{\frac{\ud^{#1} {#2}}{\ud {#3}{}^{#1}}}}
\newcommand{\psdiff}[3][{}]{{\frac{\partial^{#1}}{\partial {#3}{}^{#1}} {#2}}}
\newcommand{\sdiff}[3][{}]{{\frac{\ud^{#1}}{\ud {#3}{}^{#1}} {#2}}}
\newcommand{\tpddiff}[4][{}]{{\left(\dfrac{\partial^{#1} {#2}}{\partial {#3}{}^{#1}}\right)_{#4}}}
\newcommand{\tpsddiff}[4][{}]{{\paren{\dfrac{\partial^{#1}}{\partial {#3}{}^{#1}}{#2}}_{#4}}}
\newcommand{\pddiff}[3][{}]{{\dfrac{\partial^{#1} {#2}}{\partial {#3}{}^{#1}}}}
\newcommand{\ddiff}[3][{}]{{\dfrac{\ud^{#1} {#2}}{\ud {#3}{}^{#1}}}}
\newcommand{\psddiff}[3][{}]{{\frac{\partial^{#1}}{\partial{}^{#1} {#3}} {#2}}}
\newcommand{\sddiff}[3][{}]{{\frac{\ud^{#1}}{\ud {#3}{}^{#1}} {#2}}}
\usepackage{fancyhdr}
\usepackage{multirow}
\usepackage{fontenc}
% \usepackage{tipa}
\usepackage{ulem}
\usepackage{color}
\usepackage{cancel}
\newcommand{\hcancel}[2][black]{\setbox0=\hbox{#2}%
  \rlap{\raisebox{.45\ht0}{\textcolor{#1}{\rule{\wd0}{1pt}}}}#2}
\pagestyle{fancy}
\setlength{\headheight}{67pt}
\fancyhead{}
\fancyfoot{}
\fancyfoot[C]{\thepage}
\fancyhead[R]
{
  \psettitlename \\
  \psettitlecourse{} Problem Set \psettitlepsetnum \\
  \ifx\psettitlersection\empty
  \else
  Recitation Section \psettitlersection
  \fi
}
\renewcommand{\footruleskip}{0pt}
\renewcommand{\headrulewidth}{0.4pt}
\renewcommand{\footrulewidth}{0pt}

\newcommand\pgfmathsinandcos[3]{%
  \pgfmathsetmacro#1{sin(#3)}%
  \pgfmathsetmacro#2{cos(#3)}%
}
\newcommand\LongitudePlane[3][current plane]{%
  \pgfmathsinandcos\sinEl\cosEl{#2} % elevation
  \pgfmathsinandcos\sint\cost{#3} % azimuth
  \tikzset{#1/.estyle={cm={\cost,\sint*\sinEl,0,\cosEl,(0,0)}}}
}
\newcommand\LatitudePlane[3][current plane]{%
  \pgfmathsinandcos\sinEl\cosEl{#2} % elevation
  \pgfmathsinandcos\sint\cost{#3} % latitude
  \pgfmathsetmacro\yshift{\cosEl*\sint}
  \tikzset{#1/.estyle={cm={\cost,0,0,\cost*\sinEl,(0,\yshift)}}} %
}
\newcommand\DrawLongitudeCircle[2][1]{
  \LongitudePlane{\angEl}{#2}
  \tikzset{current plane/.prefix style={scale=#1}}
  % angle of "visibility"
  \pgfmathsetmacro\angVis{atan(sin(#2)*cos(\angEl)/sin(\angEl))} %
  \draw[current plane] (\angVis:1) arc (\angVis:\angVis+180:1);
  \draw[current plane,dashed] (\angVis-180:1) arc (\angVis-180:\angVis:1);
}
\newcommand\DrawLatitudeCircleArrow[2][1]{
  \LatitudePlane{\angEl}{#2}
  \tikzset{current plane/.prefix style={scale=#1}}
  \pgfmathsetmacro\sinVis{sin(#2)/cos(#2)*sin(\angEl)/cos(\angEl)}
  % angle of "visibility"
  \pgfmathsetmacro\angVis{asin(min(1,max(\sinVis,-1)))}
  \draw[current plane,decoration={markings, mark=at position 0.6 with {\arrow{<}}},postaction={decorate},line width=.6mm] (\angVis:1) arc (\angVis:-\angVis-180:1);
  \draw[current plane,dashed,line width=.6mm] (180-\angVis:1) arc (180-\angVis:\angVis:1);
}
\newcommand\DrawLatitudeCircle[2][1]{
  \LatitudePlane{\angEl}{#2}
  \tikzset{current plane/.prefix style={scale=#1}}
  \pgfmathsetmacro\sinVis{sin(#2)/cos(#2)*sin(\angEl)/cos(\angEl)}
  % angle of "visibility"
  \pgfmathsetmacro\angVis{asin(min(1,max(\sinVis,-1)))}
  \draw[current plane] (\angVis:1) arc (\angVis:-\angVis-180:1);
  \draw[current plane,dashed] (180-\angVis:1) arc (180-\angVis:\angVis:1);
}
\newcommand\coil[1]{
  {\rh * cos(\t * pi r)}, {\apart * (2 * #1 + \t) + \rv * sin(\t * pi r)}
}
\makeatletter
\define@key{DrawFromCenter}{style}[{->}]{
  \tikzset{DrawFromCenterPlane/.style={#1}}
}
\define@key{DrawFromCenter}{r}[1]{
  \def\@R{#1}
}
\define@key{DrawFromCenter}{center}[(0, 0)]{
  \def\@Center{#1}
}
\define@key{DrawFromCenter}{theta}[0]{
  \def\@Theta{#1}
}
\define@key{DrawFromCenter}{phi}[0]{
  \def\@Phi{#1}
}
\presetkeys{DrawFromCenter}{style, r, center, theta, phi}{}
\newcommand*\DrawFromCenter[1][]{
  \setkeys{DrawFromCenter}{#1}{
    \pgfmathsinandcos\sint\cost{\@Theta}
    \pgfmathsinandcos\sinp\cosp{\@Phi}
    \pgfmathsinandcos\sinA\cosA{\angEl}
    \pgfmathsetmacro\DX{\@R*\cost*\cosp}
    \pgfmathsetmacro\DY{\@R*(\cost*\sinp*\sinA+\sint*\cosA)}
    \draw[DrawFromCenterPlane] \@Center -- ++(\DX, \DY);
  }
}
\newcommand*\DrawFromCenterText[2][]{
  \setkeys{DrawFromCenter}{#1}{
    \pgfmathsinandcos\sint\cost{\@Theta}
    \pgfmathsinandcos\sinp\cosp{\@Phi}
    \pgfmathsinandcos\sinA\cosA{\angEl}
    \pgfmathsetmacro\DX{\@R*\cost*\cosp}
    \pgfmathsetmacro\DY{\@R*(\cost*\sinp*\sinA+\sint*\cosA)}
    \draw[DrawFromCenterPlane] \@Center -- ++(\DX, \DY) node {#2};
  }
}
\makeatother
\tikzstyle{snakearrow} = [decorate, decoration={pre length=0.2cm,
  post length=0.2cm, snake, amplitude=.4mm,
  segment length=2mm},thick, ->]
%% document-wide tikz options and styles
\tikzset{%
  >=latex, % option for nice arrows
  inner sep=0pt,%
  outer sep=2pt,%
  mark coordinate/.style={inner sep=0pt,outer sep=0pt,minimum size=3pt,
    fill=black,circle}%
}
\addtolength{\hoffset}{-1.3cm}
\addtolength{\voffset}{-2cm}
\addtolength{\textwidth}{3cm}
\addtolength{\textheight}{2.5cm}
\renewcommand{\footskip}{10pt}
\setlength{\headwidth}{\textwidth}
\setlength{\headsep}{20pt}
\setlength{\marginparwidth}{0pt}
\parindent=0pt
\psetnum{9}
\course{8.422}
\rsection{1}
\name{Yichao Yu}
\renewcommand{\thesection}{\arabic{section}.}
\renewcommand{\thesubsection}{(\alph{subsection})}
\renewcommand{\thesubsubsection}{\roman{subsubsection}.}

\begin{document}
\section{A Zeeman Slower}
\subsection{}
The maximum deceleration is achieved when the scattering rate is maximized. Since the maximum population when pumping with a laser in such a two level system is $50\%$ the maximum scattering rate is $\dfrac\gamma2$. Maximum deceleration
\eqar{
  a_{max}=&\frac{\gamma}{2}\frac{p_{rec}}{m}\\
  =&\frac{h\gamma}{2m\lambda}\\
  =&9.3\cdot10^{5}\text{m}\cdot\text{s}^{-2}
}
\subsection{}
Assuming the atom flux (and therefore the optical depth of the slower) is small so that the intensity of light is almost constant in the slower.\\
Length of the slower,
\eqar{
  L=&\frac{v_{max}^2}{2a}\\
  =&\frac{v_{max}^2}{2fa_{max}}\\
  =&\frac1f\frac{k_BT\lambda}{h\gamma}\\
  =&\frac{0.12}{f} \text{m}
  \intertext{Maximum velocity in the slower}
  v=&\sqrt{2fa_{max}\paren{L-x}}
  \intertext{Doppler shift}
  \delta_{doppler}=&\frac{v}{\lambda}\\
  =&\frac{\sqrt{2fa_{max}\paren{L-x}}}{\lambda}
  \intertext{This should be canceled by the Zeeman shift}
  B=&\frac{\sqrt{2fa_{max}\paren{L-x}}}{g\mu_B\lambda}
}

\subsection{}
Variance of $\Delta x$ for each emission for three situations
\eqar{
  \angl{\Delta p^2}_0=&p^2_{rec}\left\{
    \begin{array}{ll}
      1&\text{i}\\
      \dint_{-1}^1 x^2\ud x&\text{ii}\\
      \dint_{-1}^1 x^4\ud x&\text{iii}
    \end{array}
  \right.\\
  =&p^2_{rec}\left\{
    \begin{array}{ll}
      1&\text{i}\\
      \dfrac13&\text{ii}\\
      \dfrac15&\text{iii}
    \end{array}
  \right.
}
\eqar{
  \mathcal{D}=&\diff{\angl{\Delta p^2}}{t}\\
  =&\angl{\Delta p^2}_0\Gamma_s\\
  =&\Gamma_s p^2_{rec}\left\{
    \begin{array}{ll}
      1&\text{i}\\
      \dfrac13&\text{ii}\\
      \dfrac15&\text{iii}
    \end{array}
  \right.
}

\section{Slowing an atom with off-resonant light}
\subsection{}
Scattering rate
\eqar{
  \Gamma_s=&\frac{\gamma}{2}\frac{s}{1+s+\paren{\dfrac{2\delta}{\gamma}}^2}\\
  =&\frac{\gamma}{2}\frac{s}{1+s+\paren{\dfrac{2v}{\lambda\gamma}}^2}\\
  \diff{v}{t}=&-\frac{p_{rec}}{m}\Gamma_s\\
  =&-a_{max}\frac{s}{1+s+\paren{\dfrac{2v}{\lambda\gamma}}^2}
  \intertext{Total time}
  T=&\int_0^{v_{max}}\frac{1}{sa_{max}}\paren{1+s+\paren{\dfrac{2v}{\lambda\gamma}}^2}\ud v\\
  =&\frac{v_{max}}{sa_{max}}\paren{1+s+\dfrac{4v_{max}^2}{3\lambda^2\gamma^2}}\\
  =&43.5\frac{v_{max}}{a_{max}}\\
  =&21.9\text{ms}
}
\subsection{}
Let $t$ be the time until the atom stops
\eqar{
  t=&\frac{v}{sa_{max}}\paren{1+s+\dfrac{4v^2}{3\lambda^2\gamma^2}}\\
  \approx&\\
  v\approx&\sqrt[3]{\dfrac{3sa_{max}\lambda^2\gamma^2}{4}}t^{1/3}\\
  l\approx&\sqrt[3]{\dfrac{3sa_{max}\lambda^2\gamma^2}{4}}\int_0^{T}t^{1/3}\ud t\\
  =&\sqrt[3]{\dfrac{3sa_{max}\lambda^2\gamma^2}{4}}\frac{3}{4}T^{4/3}\\
  \approx&7.8\text{m}
}
\subsection{}
This is much longer and less efficient than a Zeeman slower.

\section{Density Limit in a MOT}
\subsection{}
The radiation from a single atom is
\eqar{
  S_{single}=&\frac{6\sigma_LI}{4\pi r^2}\hat r
  \intertext{Force}
  F_{single}=&\frac{6\sigma_L\sigma_RI}{4\pi r^2c}\hat r
  \intertext{From Gauss' law}
  \nabla\cdot F_{R}=&\frac{6\sigma_L\sigma_RIn}{c}
}
\subsection{}
For a isotropic light field, the light force is proportional to the gradient of light intensity. For the case of scattering force, each atom is acting as a source for the gradient while for attenuation force, each atom is acting as a drain for the gradient (thus the minus sign). Furthermore, the frequency of the light is the trap light so the absorption cross section should be replaced by $\sigma_L$.

\subsection{}
\eqar{
  \nabla\cdot\vec F=&\frac{6\sigma_L\paren{\sigma_R-\sigma_L}In}{c}-3\kappa\\
  n_{max}=&\frac{3\kappa c}{6\sigma_L\paren{\sigma_R-\sigma_L}I}
}

\subsection{}
On average $\sigma_R>\sigma_L$ due to the resonance component of the scattered light (Mollow triplet).

\subsection{}
This effectively replaces $n$ by $fn<n$ in $F_R$ and $F_A$ so the both $F_R$ and $F_A$ decreases (with same $n$) while $F_T$ remains the same (and might increase a little due to the lack of attenuation from the atoms). $n_{max}$ will increase by $f^{-1}$.

\end{document}
