\documentclass[10pt,fleqn]{article}
\newcommand{\name}[1]{\def\psettitlename{#1}}
\newcommand{\course}[1]{\def\psettitlecourse{#1}}
\newcommand{\rsection}[1]{\def\psettitlersection{#1}}
\newcommand{\psetnum}[1]{\def\psettitlepsetnum{#1}}
% \usepackage[journal=rsc]{chemstyle}
% \usepackage{mhchem}
\usepackage{amsmath}
\usepackage{amssymb}
\usepackage{amsfonts}
\usepackage{esint}
\usepackage{bbm}
\usepackage{amscd}
\usepackage{picinpar}
\usepackage{graphicx}
\usepackage{tikz}
\usepackage{tikz-3dplot}
\usepackage{indentfirst}
\usepackage{wrapfig}
\usepackage{units}
\usepackage{textcomp}
\usepackage[utf8x]{inputenc}
% \usepackage{feyn}
\usepackage{feynmp}
\usepackage{xkeyval}
\usepackage{xargs}
\usepackage{verbatim}
\usepackage{pgfplots}
\usetikzlibrary{
  arrows,
  calc,
  decorations.pathmorphing,
  decorations.pathreplacing,
  decorations.markings,
  fadings,
  positioning,
  shapes
}

\DeclareGraphicsRule{*}{mps}{*}{}
\newcommand{\ud}{\mathrm{d}}
\newcommand{\ue}{\mathrm{e}}
\newcommand{\ui}{\mathrm{i}}
\newcommand{\res}{\mathrm{Res}}
\newcommand{\Tr}{\mathrm{Tr}}
\newcommand{\dsum}{\displaystyle\sum}
\newcommand{\dprod}{\displaystyle\prod}
\newcommand{\dlim}{\displaystyle\lim}
\newcommand{\dint}{\displaystyle\int}
\newcommand{\fsno}[1]{{\!\not\!{#1}}}
\newcommand{\eqar}[1]
{
  \begin{align*}
    #1
  \end{align*}
}
\newcommand{\texp}[2]{\ensuremath{{#1}\times10^{#2}}}
\newcommand{\dexp}[2]{\ensuremath{{#1}\cdot10^{#2}}}
\newcommand{\eval}[2]{{\left.{#1}\right|_{#2}}}
\newcommand{\paren}[1]{{\left({#1}\right)}}
\newcommand{\lparen}[1]{{\left({#1}\right.}}
\newcommand{\rparen}[1]{{\left.{#1}\right)}}
\newcommand{\abs}[1]{{\left|{#1}\right|}}
\newcommand{\sqr}[1]{{\left[{#1}\right]}}
\newcommand{\crly}[1]{{\left\{{#1}\right\}}}
\newcommand{\angl}[1]{{\left\langle{#1}\right\rangle}}
\newcommand{\tpdiff}[4][{}]{{\paren{\frac{\partial^{#1} {#2}}{\partial {#3}{}^{#1}}}_{#4}}}
\newcommand{\tpsdiff}[4][{}]{{\paren{\frac{\partial^{#1}}{\partial {#3}{}^{#1}}{#2}}_{#4}}}
\newcommand{\pdiff}[3][{}]{{\frac{\partial^{#1} {#2}}{\partial {#3}{}^{#1}}}}
\newcommand{\diff}[3][{}]{{\frac{\ud^{#1} {#2}}{\ud {#3}{}^{#1}}}}
\newcommand{\psdiff}[3][{}]{{\frac{\partial^{#1}}{\partial {#3}{}^{#1}} {#2}}}
\newcommand{\sdiff}[3][{}]{{\frac{\ud^{#1}}{\ud {#3}{}^{#1}} {#2}}}
\newcommand{\tpddiff}[4][{}]{{\left(\dfrac{\partial^{#1} {#2}}{\partial {#3}{}^{#1}}\right)_{#4}}}
\newcommand{\tpsddiff}[4][{}]{{\paren{\dfrac{\partial^{#1}}{\partial {#3}{}^{#1}}{#2}}_{#4}}}
\newcommand{\pddiff}[3][{}]{{\dfrac{\partial^{#1} {#2}}{\partial {#3}{}^{#1}}}}
\newcommand{\ddiff}[3][{}]{{\dfrac{\ud^{#1} {#2}}{\ud {#3}{}^{#1}}}}
\newcommand{\psddiff}[3][{}]{{\frac{\partial^{#1}}{\partial{}^{#1} {#3}} {#2}}}
\newcommand{\sddiff}[3][{}]{{\frac{\ud^{#1}}{\ud {#3}{}^{#1}} {#2}}}
\usepackage{fancyhdr}
\usepackage{multirow}
\usepackage{fontenc}
% \usepackage{tipa}
\usepackage{ulem}
\usepackage{color}
\usepackage{cancel}
\newcommand{\hcancel}[2][black]{\setbox0=\hbox{#2}%
  \rlap{\raisebox{.45\ht0}{\textcolor{#1}{\rule{\wd0}{1pt}}}}#2}
\pagestyle{fancy}
\setlength{\headheight}{67pt}
\fancyhead{}
\fancyfoot{}
\fancyfoot[C]{\thepage}
\fancyhead[R]
{
  \psettitlename \\
  \psettitlecourse{} Problem Set \psettitlepsetnum \\
  \ifx\psettitlersection\empty
  \else
  Recitation Section \psettitlersection
  \fi
}
\renewcommand{\footruleskip}{0pt}
\renewcommand{\headrulewidth}{0.4pt}
\renewcommand{\footrulewidth}{0pt}

\newcommand\pgfmathsinandcos[3]{%
  \pgfmathsetmacro#1{sin(#3)}%
  \pgfmathsetmacro#2{cos(#3)}%
}
\newcommand\LongitudePlane[3][current plane]{%
  \pgfmathsinandcos\sinEl\cosEl{#2} % elevation
  \pgfmathsinandcos\sint\cost{#3} % azimuth
  \tikzset{#1/.estyle={cm={\cost,\sint*\sinEl,0,\cosEl,(0,0)}}}
}
\newcommand\LatitudePlane[3][current plane]{%
  \pgfmathsinandcos\sinEl\cosEl{#2} % elevation
  \pgfmathsinandcos\sint\cost{#3} % latitude
  \pgfmathsetmacro\yshift{\cosEl*\sint}
  \tikzset{#1/.estyle={cm={\cost,0,0,\cost*\sinEl,(0,\yshift)}}} %
}
\newcommand\DrawLongitudeCircle[2][1]{
  \LongitudePlane{\angEl}{#2}
  \tikzset{current plane/.prefix style={scale=#1}}
  % angle of "visibility"
  \pgfmathsetmacro\angVis{atan(sin(#2)*cos(\angEl)/sin(\angEl))} %
  \draw[current plane] (\angVis:1) arc (\angVis:\angVis+180:1);
  \draw[current plane,dashed] (\angVis-180:1) arc (\angVis-180:\angVis:1);
}
\newcommand\DrawLatitudeCircleArrow[2][1]{
  \LatitudePlane{\angEl}{#2}
  \tikzset{current plane/.prefix style={scale=#1}}
  \pgfmathsetmacro\sinVis{sin(#2)/cos(#2)*sin(\angEl)/cos(\angEl)}
  % angle of "visibility"
  \pgfmathsetmacro\angVis{asin(min(1,max(\sinVis,-1)))}
  \draw[current plane,decoration={markings, mark=at position 0.6 with {\arrow{<}}},postaction={decorate},line width=.6mm] (\angVis:1) arc (\angVis:-\angVis-180:1);
  \draw[current plane,dashed,line width=.6mm] (180-\angVis:1) arc (180-\angVis:\angVis:1);
}
\newcommand\DrawLatitudeCircle[2][1]{
  \LatitudePlane{\angEl}{#2}
  \tikzset{current plane/.prefix style={scale=#1}}
  \pgfmathsetmacro\sinVis{sin(#2)/cos(#2)*sin(\angEl)/cos(\angEl)}
  % angle of "visibility"
  \pgfmathsetmacro\angVis{asin(min(1,max(\sinVis,-1)))}
  \draw[current plane] (\angVis:1) arc (\angVis:-\angVis-180:1);
  \draw[current plane,dashed] (180-\angVis:1) arc (180-\angVis:\angVis:1);
}
\newcommand\coil[1]{
  {\rh * cos(\t * pi r)}, {\apart * (2 * #1 + \t) + \rv * sin(\t * pi r)}
}
\makeatletter
\define@key{DrawFromCenter}{style}[{->}]{
  \tikzset{DrawFromCenterPlane/.style={#1}}
}
\define@key{DrawFromCenter}{r}[1]{
  \def\@R{#1}
}
\define@key{DrawFromCenter}{center}[(0, 0)]{
  \def\@Center{#1}
}
\define@key{DrawFromCenter}{theta}[0]{
  \def\@Theta{#1}
}
\define@key{DrawFromCenter}{phi}[0]{
  \def\@Phi{#1}
}
\presetkeys{DrawFromCenter}{style, r, center, theta, phi}{}
\newcommand*\DrawFromCenter[1][]{
  \setkeys{DrawFromCenter}{#1}{
    \pgfmathsinandcos\sint\cost{\@Theta}
    \pgfmathsinandcos\sinp\cosp{\@Phi}
    \pgfmathsinandcos\sinA\cosA{\angEl}
    \pgfmathsetmacro\DX{\@R*\cost*\cosp}
    \pgfmathsetmacro\DY{\@R*(\cost*\sinp*\sinA+\sint*\cosA)}
    \draw[DrawFromCenterPlane] \@Center -- ++(\DX, \DY);
  }
}
\newcommand*\DrawFromCenterText[2][]{
  \setkeys{DrawFromCenter}{#1}{
    \pgfmathsinandcos\sint\cost{\@Theta}
    \pgfmathsinandcos\sinp\cosp{\@Phi}
    \pgfmathsinandcos\sinA\cosA{\angEl}
    \pgfmathsetmacro\DX{\@R*\cost*\cosp}
    \pgfmathsetmacro\DY{\@R*(\cost*\sinp*\sinA+\sint*\cosA)}
    \draw[DrawFromCenterPlane] \@Center -- ++(\DX, \DY) node {#2};
  }
}
\makeatother
\tikzstyle{snakearrow} = [decorate, decoration={pre length=0.2cm,
  post length=0.2cm, snake, amplitude=.4mm,
  segment length=2mm},thick, ->]
%% document-wide tikz options and styles
\tikzset{%
  >=latex, % option for nice arrows
  inner sep=0pt,%
  outer sep=2pt,%
  mark coordinate/.style={inner sep=0pt,outer sep=0pt,minimum size=3pt,
    fill=black,circle}%
}
\addtolength{\hoffset}{-1.3cm}
\addtolength{\voffset}{-2cm}
\addtolength{\textwidth}{3cm}
\addtolength{\textheight}{2.5cm}
\renewcommand{\footskip}{10pt}
\setlength{\headwidth}{\textwidth}
\setlength{\headsep}{20pt}
\setlength{\marginparwidth}{0pt}
\parindent=0pt
\psetnum{8}
\course{8.422}
\rsection{1}
\name{Yichao Yu}
\renewcommand{\thesection}{\arabic{section}.}
\renewcommand{\thesubsection}{(\alph{subsection})}
\renewcommand{\thesubsubsection}{\roman{subsubsection}.}

\begin{document}
\section{Classical Model of the Light Force}
\subsection{Time averaged force}
\eqar{
  \angl{\vec F}=&\angl{\paren{\hat p\cdot\hat\epsilon}\paren{u\cos\paren{\omega t+\theta}-v\sin\paren{\omega t+\theta}}\nabla\paren{E_0\cos\paren{\omega t+\theta}}}\\
  =&\angl{\paren{\hat p\cdot\hat\epsilon}\paren{u\cos\paren{\omega t+\theta}-v\sin\paren{\omega t+\theta}}\paren{\cos\paren{\omega t+\theta}\nabla E_0-E_0\sin\paren{\omega t+\theta}\nabla\theta}}\\
  =&\frac12\paren{\hat p\cdot\hat\epsilon}\paren{u\nabla E_0+vE_0\nabla\theta}
}
\subsection{The potential picture}
\eqar{
  \angl{U}=&-\angl{\vec p\cdot\vec E}\\
  =&-\angl{\paren{\hat p\cdot\hat\epsilon}\paren{u\cos\paren{\omega t+\theta}-v\sin\paren{\omega t+\theta}}E_0\cos\paren{\omega t+\theta}}\\
  =&-\frac12\paren{\hat p\cdot\hat\epsilon}uE_0\\
  \angl{\vec F}=&\frac12\paren{\hat p\cdot\hat\epsilon}u\nabla E_0
}
The potential picture includes an extra term from the change of dipole moment when the dipole moves.

\subsection{Dipole moment of electron}
Let $\vec r=\hat\epsilon \tilde r_0\ue^{\ui\omega t}$
\eqar{
  &-e\hat\epsilon E_0\ue^{\ui\omega t+\ui\theta}\\
  =&\paren{-m\omega^2+\ui\omega\gamma+m\omega_0^2}\hat\epsilon\tilde r_0\ue^{\ui\omega t}\\
  \tilde r_0=&\frac{e E_0\ue^{\ui\theta}}{m\omega^2-\ui\omega\gamma-m\omega_0^2}\\
  \tilde{\vec p}=&-\hat\epsilon\ue^{\ui\omega t}\frac{e^2E_0\ue^{\ui\theta}}{m\omega^2-\ui\omega\gamma-m\omega_0^2}
  \intertext{Real part}
  \vec p=&-\hat\epsilon\Re\paren{\ue^{\ui\omega t+\ui\theta}\frac{e^2E_0}{m\omega^2-\ui\omega\gamma-m\omega_0^2}}\\
  =&-\hat\epsilon\paren{
    \cos\paren{\omega t+\theta}\Re\paren{\frac{e^2E_0}{m\omega^2-\ui\omega\gamma-m\omega_0^2}}
    -\sin\paren{\omega t+\theta}\Im\paren{\frac{e^2E_0}{m\omega^2-\ui\omega\gamma-m\omega_0^2}}
  }
  \intertext{Where}
  &\frac{e^2E_0}{m\omega^2-\ui\omega\gamma-m\omega_0^2}\\
  =&e^2E_0\frac{1}{m\paren{\omega^2-\omega_0^2}-\ui\omega\gamma}\\
  \approx&\frac{e^2E_0}{m\omega_0}\frac{1}{2\delta-\ui\Gamma}\\
  =&\frac{e^2E_0}{m\omega_0}\frac{2\delta+\ui\Gamma}{4\delta^2+\Gamma^2}
  \intertext{Compare to the definition of $u$ and $v$}
  u=&-\Re\paren{\frac{e^2E_0}{m\omega^2-\ui\omega\gamma-m\omega_0^2}}\\
  =&-\frac{e^2E_0}{m\omega_0}\frac{2\delta}{4\delta^2+\Gamma^2}\\
  v=&-\Im\paren{\frac{e^2E_0}{m\omega^2-\ui\omega\gamma-m\omega_0^2}}\\
  =&-\frac{e^2E_0}{m\omega_0}\frac{\Gamma}{4\delta^2+\Gamma^2}
  \intertext{Force}
  \angl{\vec F}=&-\frac{e^2}{2m\omega_0}\frac{2E_0\nabla E_0\delta+\Gamma E_0^2\nabla\theta}{4\delta^2+\Gamma^2}\\
  =&-\frac{e^2}{2m\omega_0}\frac{\delta\nabla E_0^2+\Gamma E_0^2\nabla\theta}{4\delta^2+\Gamma^2}
}

\subsection{Force on a two-level atom}
Since $\omega_R\propto E_0$
\eqar{
  \angl{\vec F}\propto&-\frac{\delta\nabla \omega_R^2+\Gamma \omega_R^2\nabla\theta}{4\delta^2+\Gamma^2}\\
  \propto&F_{quantum}
}

\section{Master equation for a damped optical cavity}
\subsection{}
\eqar{
  \langle\tilde\psi_1|\tilde\psi_1\rangle=&\Gamma\ud t\langle\psi|a^\dagger a|\psi\rangle\\
  =&\ud p
  \intertext{Since the imaginary and real part of $H$ commute}
  \langle\tilde\psi_0|\tilde\psi_0\rangle=&\langle\psi|\ue^{\ui H^\dagger\ud t/\hbar}\ue^{-\ui H\ud t/\hbar}|\psi\rangle\\
  =&\langle\psi|\ue^{-\Gamma \ud t a^\dagger a}|\psi\rangle\\
  =&\exp\paren{-\Gamma \ud t \langle\psi|a^\dagger a|\psi\rangle}\\
  =&\ue^{-\ud p}\\
  =&1-\ud p
  \intertext{Therefore}
  |\tilde\psi_1\rangle=&\frac{|\tilde\psi_1\rangle}{\ud p}\\
  |\tilde\psi_0\rangle=&\frac{|\tilde\psi_0\rangle}{1-\ud p}
}
The lossy term in $H$ is proportional to $a^\dagger a$ because the loss is proportional to $a^\dagger a$. The prefactor gives the correct overall normalization.

\subsection{}
\eqar{
  \rho+\ud\rho=&|\tilde\psi_1\rangle\langle\tilde\psi_1|+|\tilde\psi_0\rangle\langle\tilde\psi_0|\\
  =&\Gamma\ud ta|\psi\rangle\langle\psi|a^\dagger+\paren{1-\frac{\ui H^\dagger\ud t}{\hbar}}|\psi\rangle\langle\psi|\paren{1+\frac{\ui H\ud t}{\hbar}}\\
  =&\rho+\Gamma\ud ta\rho a^\dagger+\rho\frac{\ui H^\dagger\ud t}{\hbar}-\frac{\ui H\ud t}{\hbar}\rho\\
  =&\rho+\Gamma\ud ta\rho a^\dagger
  -\frac{\ui\ud t}{\hbar}\sqr{H_0, \rho}
  -\ud t\frac{\Gamma}{2}\rho a^\dagger a-\ud t\frac{\Gamma}{2}a^\dagger a\rho
}
\subsection{}
\eqar{
  \diff{\rho}{t}=&\Gamma a\rho a^\dagger-\frac{\ui}{\hbar}\sqr{H_0, \rho}
  -\frac{\Gamma}{2}\rho a^\dagger a-\frac{\Gamma}{2}a^\dagger a\rho\\
  =&-\frac{\ui}{\hbar}\sqr{H_0, \rho}
  -\frac{1}{2}\paren{\Gamma\rho a^\dagger a-2\Gamma a\rho a^\dagger
    +\Gamma a^\dagger a\rho}\\
  =&-\frac{\ui}{\hbar}\sqr{H_0, \rho}
  -\frac{1}{2}\paren{\rho C^\dagger C-2C\rho C^\dagger
    +C^\dagger C\rho}
}
where $C=\sqrt\Gamma a$

\end{document}
