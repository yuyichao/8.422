\documentclass[10pt,fleqn]{article}
\newcommand{\name}[1]{\def\psettitlename{#1}}
\newcommand{\course}[1]{\def\psettitlecourse{#1}}
\newcommand{\rsection}[1]{\def\psettitlersection{#1}}
\newcommand{\psetnum}[1]{\def\psettitlepsetnum{#1}}
% \usepackage[journal=rsc]{chemstyle}
% \usepackage{mhchem}
\usepackage{amsmath}
\usepackage{amssymb}
\usepackage{amsfonts}
\usepackage{esint}
\usepackage{bbm}
\usepackage{amscd}
\usepackage{picinpar}
\usepackage{graphicx}
\usepackage{tikz}
\usepackage{tikz-3dplot}
\usepackage{indentfirst}
\usepackage{wrapfig}
\usepackage{units}
\usepackage{textcomp}
\usepackage[utf8x]{inputenc}
% \usepackage{feyn}
\usepackage{feynmp}
\usepackage{xkeyval}
\usepackage{xargs}
\usepackage{verbatim}
\usepackage{pgfplots}
\usetikzlibrary{
  arrows,
  calc,
  decorations.pathmorphing,
  decorations.pathreplacing,
  decorations.markings,
  fadings,
  positioning,
  shapes
}

\DeclareGraphicsRule{*}{mps}{*}{}
\newcommand{\ud}{\mathrm{d}}
\newcommand{\ue}{\mathrm{e}}
\newcommand{\ui}{\mathrm{i}}
\newcommand{\res}{\mathrm{Res}}
\newcommand{\Tr}{\mathrm{Tr}}
\newcommand{\dsum}{\displaystyle\sum}
\newcommand{\dprod}{\displaystyle\prod}
\newcommand{\dlim}{\displaystyle\lim}
\newcommand{\dint}{\displaystyle\int}
\newcommand{\fsno}[1]{{\!\not\!{#1}}}
\newcommand{\eqar}[1]
{
  \begin{align*}
    #1
  \end{align*}
}
\newcommand{\texp}[2]{\ensuremath{{#1}\times10^{#2}}}
\newcommand{\dexp}[2]{\ensuremath{{#1}\cdot10^{#2}}}
\newcommand{\eval}[2]{{\left.{#1}\right|_{#2}}}
\newcommand{\paren}[1]{{\left({#1}\right)}}
\newcommand{\lparen}[1]{{\left({#1}\right.}}
\newcommand{\rparen}[1]{{\left.{#1}\right)}}
\newcommand{\abs}[1]{{\left|{#1}\right|}}
\newcommand{\sqr}[1]{{\left[{#1}\right]}}
\newcommand{\crly}[1]{{\left\{{#1}\right\}}}
\newcommand{\angl}[1]{{\left\langle{#1}\right\rangle}}
\newcommand{\tpdiff}[4][{}]{{\paren{\frac{\partial^{#1} {#2}}{\partial {#3}{}^{#1}}}_{#4}}}
\newcommand{\tpsdiff}[4][{}]{{\paren{\frac{\partial^{#1}}{\partial {#3}{}^{#1}}{#2}}_{#4}}}
\newcommand{\pdiff}[3][{}]{{\frac{\partial^{#1} {#2}}{\partial {#3}{}^{#1}}}}
\newcommand{\diff}[3][{}]{{\frac{\ud^{#1} {#2}}{\ud {#3}{}^{#1}}}}
\newcommand{\psdiff}[3][{}]{{\frac{\partial^{#1}}{\partial {#3}{}^{#1}} {#2}}}
\newcommand{\sdiff}[3][{}]{{\frac{\ud^{#1}}{\ud {#3}{}^{#1}} {#2}}}
\newcommand{\tpddiff}[4][{}]{{\left(\dfrac{\partial^{#1} {#2}}{\partial {#3}{}^{#1}}\right)_{#4}}}
\newcommand{\tpsddiff}[4][{}]{{\paren{\dfrac{\partial^{#1}}{\partial {#3}{}^{#1}}{#2}}_{#4}}}
\newcommand{\pddiff}[3][{}]{{\dfrac{\partial^{#1} {#2}}{\partial {#3}{}^{#1}}}}
\newcommand{\ddiff}[3][{}]{{\dfrac{\ud^{#1} {#2}}{\ud {#3}{}^{#1}}}}
\newcommand{\psddiff}[3][{}]{{\frac{\partial^{#1}}{\partial{}^{#1} {#3}} {#2}}}
\newcommand{\sddiff}[3][{}]{{\frac{\ud^{#1}}{\ud {#3}{}^{#1}} {#2}}}
\usepackage{fancyhdr}
\usepackage{multirow}
\usepackage{fontenc}
% \usepackage{tipa}
\usepackage{ulem}
\usepackage{color}
\usepackage{cancel}
\newcommand{\hcancel}[2][black]{\setbox0=\hbox{#2}%
  \rlap{\raisebox{.45\ht0}{\textcolor{#1}{\rule{\wd0}{1pt}}}}#2}
\pagestyle{fancy}
\setlength{\headheight}{67pt}
\fancyhead{}
\fancyfoot{}
\fancyfoot[C]{\thepage}
\fancyhead[R]
{
  \psettitlename \\
  \psettitlecourse{} Problem Set \psettitlepsetnum \\
  \ifx\psettitlersection\empty
  \else
  Recitation Section \psettitlersection
  \fi
}
\renewcommand{\footruleskip}{0pt}
\renewcommand{\headrulewidth}{0.4pt}
\renewcommand{\footrulewidth}{0pt}

\newcommand\pgfmathsinandcos[3]{%
  \pgfmathsetmacro#1{sin(#3)}%
  \pgfmathsetmacro#2{cos(#3)}%
}
\newcommand\LongitudePlane[3][current plane]{%
  \pgfmathsinandcos\sinEl\cosEl{#2} % elevation
  \pgfmathsinandcos\sint\cost{#3} % azimuth
  \tikzset{#1/.estyle={cm={\cost,\sint*\sinEl,0,\cosEl,(0,0)}}}
}
\newcommand\LatitudePlane[3][current plane]{%
  \pgfmathsinandcos\sinEl\cosEl{#2} % elevation
  \pgfmathsinandcos\sint\cost{#3} % latitude
  \pgfmathsetmacro\yshift{\cosEl*\sint}
  \tikzset{#1/.estyle={cm={\cost,0,0,\cost*\sinEl,(0,\yshift)}}} %
}
\newcommand\DrawLongitudeCircle[2][1]{
  \LongitudePlane{\angEl}{#2}
  \tikzset{current plane/.prefix style={scale=#1}}
  % angle of "visibility"
  \pgfmathsetmacro\angVis{atan(sin(#2)*cos(\angEl)/sin(\angEl))} %
  \draw[current plane] (\angVis:1) arc (\angVis:\angVis+180:1);
  \draw[current plane,dashed] (\angVis-180:1) arc (\angVis-180:\angVis:1);
}
\newcommand\DrawLatitudeCircleArrow[2][1]{
  \LatitudePlane{\angEl}{#2}
  \tikzset{current plane/.prefix style={scale=#1}}
  \pgfmathsetmacro\sinVis{sin(#2)/cos(#2)*sin(\angEl)/cos(\angEl)}
  % angle of "visibility"
  \pgfmathsetmacro\angVis{asin(min(1,max(\sinVis,-1)))}
  \draw[current plane,decoration={markings, mark=at position 0.6 with {\arrow{<}}},postaction={decorate},line width=.6mm] (\angVis:1) arc (\angVis:-\angVis-180:1);
  \draw[current plane,dashed,line width=.6mm] (180-\angVis:1) arc (180-\angVis:\angVis:1);
}
\newcommand\DrawLatitudeCircle[2][1]{
  \LatitudePlane{\angEl}{#2}
  \tikzset{current plane/.prefix style={scale=#1}}
  \pgfmathsetmacro\sinVis{sin(#2)/cos(#2)*sin(\angEl)/cos(\angEl)}
  % angle of "visibility"
  \pgfmathsetmacro\angVis{asin(min(1,max(\sinVis,-1)))}
  \draw[current plane] (\angVis:1) arc (\angVis:-\angVis-180:1);
  \draw[current plane,dashed] (180-\angVis:1) arc (180-\angVis:\angVis:1);
}
\newcommand\coil[1]{
  {\rh * cos(\t * pi r)}, {\apart * (2 * #1 + \t) + \rv * sin(\t * pi r)}
}
\makeatletter
\define@key{DrawFromCenter}{style}[{->}]{
  \tikzset{DrawFromCenterPlane/.style={#1}}
}
\define@key{DrawFromCenter}{r}[1]{
  \def\@R{#1}
}
\define@key{DrawFromCenter}{center}[(0, 0)]{
  \def\@Center{#1}
}
\define@key{DrawFromCenter}{theta}[0]{
  \def\@Theta{#1}
}
\define@key{DrawFromCenter}{phi}[0]{
  \def\@Phi{#1}
}
\presetkeys{DrawFromCenter}{style, r, center, theta, phi}{}
\newcommand*\DrawFromCenter[1][]{
  \setkeys{DrawFromCenter}{#1}{
    \pgfmathsinandcos\sint\cost{\@Theta}
    \pgfmathsinandcos\sinp\cosp{\@Phi}
    \pgfmathsinandcos\sinA\cosA{\angEl}
    \pgfmathsetmacro\DX{\@R*\cost*\cosp}
    \pgfmathsetmacro\DY{\@R*(\cost*\sinp*\sinA+\sint*\cosA)}
    \draw[DrawFromCenterPlane] \@Center -- ++(\DX, \DY);
  }
}
\newcommand*\DrawFromCenterText[2][]{
  \setkeys{DrawFromCenter}{#1}{
    \pgfmathsinandcos\sint\cost{\@Theta}
    \pgfmathsinandcos\sinp\cosp{\@Phi}
    \pgfmathsinandcos\sinA\cosA{\angEl}
    \pgfmathsetmacro\DX{\@R*\cost*\cosp}
    \pgfmathsetmacro\DY{\@R*(\cost*\sinp*\sinA+\sint*\cosA)}
    \draw[DrawFromCenterPlane] \@Center -- ++(\DX, \DY) node {#2};
  }
}
\makeatother
\tikzstyle{snakearrow} = [decorate, decoration={pre length=0.2cm,
  post length=0.2cm, snake, amplitude=.4mm,
  segment length=2mm},thick, ->]
%% document-wide tikz options and styles
\tikzset{%
  >=latex, % option for nice arrows
  inner sep=0pt,%
  outer sep=2pt,%
  mark coordinate/.style={inner sep=0pt,outer sep=0pt,minimum size=3pt,
    fill=black,circle}%
}
\addtolength{\hoffset}{-1.3cm}
\addtolength{\voffset}{-2cm}
\addtolength{\textwidth}{3cm}
\addtolength{\textheight}{2.5cm}
\renewcommand{\footskip}{10pt}
\setlength{\headwidth}{\textwidth}
\setlength{\headsep}{20pt}
\setlength{\marginparwidth}{0pt}
\parindent=0pt
\psetnum{7}
\course{8.422}
\rsection{1}
\name{Yichao Yu}
\renewcommand{\thesection}{\arabic{section}.}
\renewcommand{\thesubsection}{(\alph{subsection})}
\renewcommand{\thesubsubsection}{\roman{subsubsection}.}

\usepackage{pst-solides3d}
\begin{document}
\section{}
\subsection{}
(Shouldn't the condition be $\Omega\ll\delta$ instead of $\Omega\ll\Gamma$?)\\
Let $\delta=\paren{\omega_0-\omega_L}$
\eqar{
  \rho_{ee}=&\rho'_{ee}\ue^{-\Gamma t}\\
  \rho_{ge}=&\rho'_{ge}\ue^{-\Gamma t / 2+\ui\delta t}\\
  \dot\rho'_{ee}=&\ui\frac{\Omega\ue^{\Gamma t/2}}{2}\paren{{\rho'_{ge}}^*\ue^{-\ui\delta t}-\rho'_{ge}\ue^{\ui\delta t}}\\
  \dot\rho'_{ge}=&-\ui\frac{\Omega\ue^{\Gamma t/2}}{2}\paren{2\rho'_{ee}\ue^{-\Gamma t}-1}\ue^{-\ui\delta t}
  \intertext{$0$th order in $\Omega$}
  \rho'_{ee0}=&0\\
  \rho'_{ge0}=&0
  \intertext{$1$st order in $\Omega$}
  \rho'_{ee1}=&0\\
  \dot\rho'_{ge1}=&\ui\frac{\Omega}{2}\ue^{\Gamma t/2-\ui\delta t}\\
  \rho'_{ge1}=&\ui\frac{\Omega}{2\paren{\Gamma/2-\ui\delta}}\paren{\ue^{\Gamma t/2-\ui\delta t}-1}\\
  {\rho'_{ge1}}^*=&-\ui\frac{\Omega}{2\paren{\Gamma/2+\ui\delta}}\paren{\ue^{\Gamma t/2+\ui\delta t}-1}
  \intertext{$2$nd order in $\Omega$}
  \dot\rho'_{ee2}=&\ui\frac{\Omega\ue^{\Gamma t/2}}{2}\paren{{\rho'_{ge1}}^*\ue^{-\ui\delta t}-\rho'_{ge1}\ue^{\ui\delta t}}\\
  =&\ui\frac{\Omega\ue^{\Gamma t/2}}{2}\paren{-\ui\frac{\Omega}{2\paren{\Gamma/2+\ui\delta}}\paren{\ue^{\Gamma t/2+\ui\delta t}-1}\ue^{-\ui\delta t}-\ui\frac{\Omega}{2\paren{\Gamma/2-\ui\delta}}\paren{\ue^{\Gamma t/2-\ui\delta t}-1}\ue^{\ui\delta t}}\\
  =&\frac{\Omega^2}{4}\paren{\frac{1}{\Gamma/2+\ui\delta}\paren{\ue^{\Gamma t}-\ue^{\Gamma t/2-\ui\delta t}}+\frac{1}{\Gamma/2-\ui\delta}\paren{\ue^{\Gamma t}-\ue^{\Gamma t/2+\ui\delta t}}}\\
  \rho'_{ee2}=&C_0+\frac{\Omega^2}{4}\paren{\frac{1}{\Gamma/2+\ui\delta}\paren{\frac{1}{\Gamma}\ue^{\Gamma t}-\frac{1}{\Gamma/2-\ui\delta}\ue^{\Gamma t/2-\ui\delta t}}+\frac{1}{\Gamma/2-\ui\delta}\paren{\frac{1}{\Gamma}\ue^{\Gamma t}-\frac{1}{\Gamma/2+\ui\delta}\ue^{\Gamma t/2+\ui\delta t}}}\\
  =&C_0+\frac{\Omega^2}{\Gamma^2+4\delta^2}\paren{\frac{\Gamma/2-\ui\delta}{\Gamma}\ue^{\Gamma t}-\ue^{\Gamma t/2-\ui\delta t}+\frac{\Gamma/2+\ui\delta}{\Gamma}\ue^{\Gamma t}-\ue^{\Gamma t/2+\ui\delta t}}\\
  =&C_0+\frac{\Omega^2}{\Gamma^2+4\delta^2}\paren{\ue^{\Gamma t}-2\cos\paren{\delta t}\ue^{\Gamma t/2}}
  \intertext{Since $\rho'(0)=0$}
  \rho'_{ee2}=&\frac{\Omega^2}{\Gamma^2+4\delta^2}\paren{1+\ue^{\Gamma t}-2\cos\paren{\delta t}\ue^{\Gamma t/2}}
  \intertext{Therefore, to the second order in $\Omega$}
  \rho_{ee}=&\frac{\Omega^2}{\Gamma^2+4\delta^2}\paren{1+\ue^{-\Gamma t}-2\cos\paren{\delta t}\ue^{-\Gamma t/2}}
}
In the limit of $\Gamma\rightarrow0$, this turns into a Rabi flopping at the frequency of the detuning.
\subsection{}
Expand the solution to second order in $t$
\eqar{
  \rho_{ee}=&\frac{\Omega^2}{\Gamma^2+4\delta^2}\paren{1+1-\Gamma t+\frac{\Gamma^2t^2}{2}-2\paren{1-\frac{\delta^2t^2}{2}}\paren{1-\frac{\Gamma t}{2}+\frac{\Gamma^2t^2}{8}}}\\
  =&\frac{\Omega^2}{\Gamma^2+4\delta^2}\paren{1+1-\Gamma t+\frac{\Gamma^2t^2}{2}-2+\delta^2t^2+\Gamma t-\frac{\Gamma^2t^2}{4}}\\
  =&\frac{\Omega^2}{\Gamma^2+4\delta^2}\paren{\frac{\Gamma^2t^2}{4}+\delta^2t^2}\\
  =&\frac{\Omega^2t^2}{4}
}
When the pulse is very short, the decay haven't started yet and the atom also doesn't have enough time to figure out that the frequency is wrong.

\section{}
\subsection{}
In the $|0g\rangle$ subspace, the Hamiltonian is $\dfrac{\hbar}{2}\paren{\delta-2\omega_0}$ and in the $|1g\rangle$, $|0e\rangle$ subspace, the Hamiltonian is $\dfrac{\hbar}{2}\paren{\Omega_1\sigma_x-\delta\sigma_z}$
\eqar{
  \ue^{-\ui Ht/\hbar}=&\ue^{\ui\paren{\delta/2+\omega_0}t}\oplus\ue^{-\ui\paren{\Omega_1\sigma_x-\delta\sigma_z}t/2}\\
  =&\ue^{-\ui\paren{\delta/2+\omega_0}t}\oplus
  \paren{\cos\frac{\Omega t}{2}-\ui\paren{\frac{\Omega_1}{\Omega}\sigma_x-\frac{\delta}{\Omega}\sigma_z}\sin\frac{\Omega t}{2}}\\
  =&\ue^{\ui\paren{\delta/2+\omega_0}t}|0g\rangle\langle0g|+\paren{\cos\frac{\Omega t}{2}+\ui\frac{\delta}{\Omega}\sin\frac{\Omega t}{2}}|0e\rangle\langle0e|+\paren{\cos\frac{\Omega t}{2}-\ui\frac{\delta}{\Omega}\sin\frac{\Omega t}{2}}|1g\rangle\langle1g|\\
  &-\ui\frac{\Omega_1}{\Omega}\sin\frac{\Omega t}{2}\paren{|0e\rangle\langle1g|+|1g\rangle\langle0e|}
}
\subsection{}
Since the system is either in $|0e\rangle$ or $|1g\rangle$ the atom is in state $e$ if there's no photon in the cavity and $g$ if there's one photon in the cavity.
\subsection{}
In the same subspace witht the first problem,
\eqar{
  \rho_0=&\begin{pmatrix}
    \rho_{gg}&\rho_{ge}\\
    \rho_{eg}&\rho_{ee}\\
    &&0
  \end{pmatrix}\\
  U=&\begin{pmatrix}
    \ue^{\ui\paren{\delta/2+\omega_0}t}&&\\
    &\cos\dfrac{\Omega t}{2}+\ui\dfrac{\delta}{\Omega}\sin\dfrac{\Omega t}{2}&-\ui\dfrac{\Omega_1}{\Omega}\sin\dfrac{\Omega t}{2}\\
    &-\ui\dfrac{\Omega_1}{\Omega}\sin\dfrac{\Omega t}{2}&\cos\dfrac{\Omega t}{2}-\ui\dfrac{\delta}{\Omega}\sin\dfrac{\Omega t}{2}
  \end{pmatrix}\\
  \rho_t=&U^\dagger\rho_0U\\
  =&\begin{pmatrix}
    \ue^{\ui\paren{\delta/2+\omega_0}t}&&\\
    &\cos\dfrac{\Omega t}{2}+\ui\dfrac{\delta}{\Omega}\sin\dfrac{\Omega t}{2}&-\ui\dfrac{\Omega_1}{\Omega}\sin\dfrac{\Omega t}{2}\\
    &-\ui\dfrac{\Omega_1}{\Omega}\sin\dfrac{\Omega t}{2}&\cos\dfrac{\Omega t}{2}-\ui\dfrac{\delta}{\Omega}\sin\dfrac{\Omega t}{2}
  \end{pmatrix}\begin{pmatrix}
    \rho_{gg}&\rho_{ge}\\
    \rho_{eg}&\rho_{ee}\\
    &&0
  \end{pmatrix}\\
  &\begin{pmatrix}
    \ue^{-\ui\paren{\delta/2+\omega_0}t}&&\\
    &\cos\dfrac{\Omega t}{2}-\ui\dfrac{\delta}{\Omega}\sin\dfrac{\Omega t}{2}&+\ui\dfrac{\Omega_1}{\Omega}\sin\dfrac{\Omega t}{2}\\
    &\ui\dfrac{\Omega_1}{\Omega}\sin\dfrac{\Omega t}{2}&\cos\dfrac{\Omega t}{2}+\ui\dfrac{\delta}{\Omega}\sin\dfrac{\Omega t}{2}
  \end{pmatrix}\\
  =&\begin{pmatrix}
    \rho_{gg}\ue^{\ui\paren{\delta/2+\omega_0}t}&\rho_{ge}\ue^{\ui\paren{\delta/2+\omega_0}t}&\\
    \rho_{eg}\paren{\cos\dfrac{\Omega t}{2}+\ui\dfrac{\delta}{\Omega}\sin\dfrac{\Omega t}{2}}&\rho_{ee}\paren{\cos\dfrac{\Omega t}{2}+\ui\dfrac{\delta}{\Omega}\sin\dfrac{\Omega t}{2}}\\
    -\rho_{eg}\ui\dfrac{\Omega_1}{\Omega}\sin\dfrac{\Omega t}{2}&-\rho_{ee}\ui\dfrac{\Omega_1}{\Omega}\sin\dfrac{\Omega t}{2}&0
  \end{pmatrix}\\
  &\begin{pmatrix}
    \ue^{-\ui\paren{\delta/2+\omega_0}t}&&\\
    &\cos\dfrac{\Omega t}{2}-\ui\dfrac{\delta}{\Omega}\sin\dfrac{\Omega t}{2}&+\ui\dfrac{\Omega_1}{\Omega}\sin\dfrac{\Omega t}{2}\\
    &\ui\dfrac{\Omega_1}{\Omega}\sin\dfrac{\Omega t}{2}&\cos\dfrac{\Omega t}{2}+\ui\dfrac{\delta}{\Omega}\sin\dfrac{\Omega t}{2}
  \end{pmatrix}
  \intertext{Let $\rho_{ge}'=\rho_{ge}\ue^{\ui\paren{\delta/2+\omega_0}t}$ and $\rho_{eg}'=\rho_{eg}\ue^{-\ui\paren{\delta/2+\omega_0}t}$}
  \rho=&\begin{pmatrix}
    \rho_{gg}&\rho'_{ge}\paren{\cos\frac{\Omega t}{2}-\ui\frac{\delta}{\Omega}\sin\frac{\Omega t}{2}}&\rho'_{ge}\ui\frac{\Omega_1}{\Omega}\sin\frac{\Omega t}{2}\\
    \rho'_{eg}\paren{\cos\frac{\Omega t}{2}+\ui\frac{\delta}{\Omega}\sin\frac{\Omega t}{2}}&\rho_{ee}\paren{\cos^2\frac{\Omega t}{2}+\frac{\delta^2}{\Omega^2}\sin^2\frac{\Omega t}{2}}&\rho_{ee}\paren{\cos\frac{\Omega t}{2}+\ui\frac{\delta}{\Omega}\sin\frac{\Omega t}{2}}\ui\frac{\Omega_1}{\Omega}\sin\frac{\Omega t}{2}\\
    -\rho'_{eg}\ui\frac{\Omega_1}{\Omega}\sin\frac{\Omega t}{2}&-\rho_{ee}\ui\frac{\Omega_1}{\Omega}\sin\frac{\Omega t}{2}\paren{\cos\frac{\Omega t}{2}-\ui\frac{\delta}{\Omega}\sin\frac{\Omega t}{2}}&\rho_{ee}\frac{\Omega_1^2}{\Omega^2}\sin^2\frac{\Omega t}{2}
  \end{pmatrix}
}
The density matrix of the atom is
\eqar{
  \rho_{atom}=&\begin{pmatrix}
    \rho_{gg}+\rho_{ee}\dfrac{\Omega_1^2}{\Omega^2}\sin^2\dfrac{\Omega t}{2}&\rho_{ge}\ue^{\ui\paren{\delta/2+\omega_0}t}\paren{\cos\dfrac{\Omega t}{2}-\ui\dfrac{\delta}{\Omega}\sin\dfrac{\Omega t}{2}}\\
    \rho_{eg}\ue^{-\ui\paren{\delta/2+\omega_0}t}\paren{\cos\dfrac{\Omega t}{2}+\ui\dfrac{\delta}{\Omega}\sin\dfrac{\Omega t}{2}}&\rho_{ee}\paren{\cos^2\dfrac{\Omega t}{2}+\dfrac{\delta^2}{\Omega^2}\sin^2\dfrac{\Omega t}{2}}
  \end{pmatrix}
}
\subsection{}
\eqar{
  \rho_{atom}\paren{\Delta t}\approx&\begin{pmatrix}
    \rho_{gg}+\rho_{ee}\dfrac{\Omega_1^2\Delta t^2}{4}&\rho_{ge}\ue^{\ui\paren{\delta/2+\omega_0}\Delta t}\paren{1-\dfrac{\Omega^2\Delta t^2}{8}-\ui\dfrac{\delta \Delta t}{2}}\\
    \rho_{eg}\ue^{-\ui\paren{\delta/2+\omega_0}\Delta t}\paren{1-\dfrac{\Omega^2\Delta t^2}{8}+\ui\dfrac{\delta \Delta t}{2}}&\rho_{ee}\paren{1-\dfrac{\Omega_1^2\Delta t^2}{4}}
  \end{pmatrix}\\
  \rho'_{atom}\paren{t}\approx&\begin{pmatrix}
    1-\rho_{ee}'\paren{t}&\rho_{ge}\ue^{\ui\paren{\delta/2+\omega_0}t}\paren{1-\dfrac{\Omega^2\Delta t^2}{8}-\ui\dfrac{\delta \Delta t}{2}}^{t/\Delta t}\\
    \rho_{eg}\ue^{-\ui\paren{\delta/2+\omega_0}t}\paren{1-\dfrac{\Omega^2\Delta t^2}{8}+\ui\dfrac{\delta \Delta t}{2}}^{t/\Delta t}&\rho_{ee}\paren{1-\dfrac{\Omega_1^2\Delta t^2}{4}}^{t/\Delta t}
  \end{pmatrix}\\
  \approx&\begin{pmatrix}
    1-\rho_{ee}\exp\paren{-\dfrac{\Omega_1^2\Delta t t}{4}}&\rho_{ge}\ue^{\ui\omega_0t}\exp\paren{-\dfrac{\Omega_1^2\Delta t t}{8}}\\
    \rho_{eg}\ue^{-\ui\omega_0t}\exp\paren{-\dfrac{\Omega_1^2\Delta t t}{8}}&\rho_{ee}\exp\paren{-\dfrac{\Omega_1^2\Delta t t}{4}}
  \end{pmatrix}\\
  =&\begin{pmatrix}
    1-\rho_{ee}\ue^{-\Gamma t}&\rho_{ge}\ue^{\ui\omega_0t}\exp\paren{-\dfrac{\Gamma t}{2}}\\
    \rho_{eg}\ue^{-\ui\omega_0t}\exp\paren{-\dfrac{\Gamma t}{2}}&\rho_{ee}\ue^{-\Gamma t}
  \end{pmatrix}
}
where $\Gamma=\dfrac{\Omega_1^2\Delta t}{4}$
\subsection{}
On the Bloch sphere
\eqar{
  \vec r=&\begin{pmatrix}
    2\rho_{ge}\cos{\omega_0t}\exp\paren{-\dfrac{\Gamma t}{2}}\\
    -2\rho_{ge}\sin{\omega_0t}\exp\paren{-\dfrac{\Gamma t}{2}}\\
    2\rho_{ee}\ue^{-\Gamma t}-1
  \end{pmatrix}
  \intertext{For $\paren{|e\rangle+|g\rangle}/\sqrt2$}
  \rho_{ee}=&\frac12\\
  \rho_{ge}=&\frac12\\
  \vec r=&\begin{pmatrix}
    \cos{\omega_0t}\exp\paren{-\dfrac{\Gamma t}{2}}\\
    -\sin{\omega_0t}\exp\paren{-\dfrac{\Gamma t}{2}}\\
    \ue^{-\Gamma t}-1
  \end{pmatrix}
  \intertext{For $\paren{|e\rangle-|g\rangle}/\sqrt2$}
  \rho_{ee}=&\frac12\\
  \rho_{ge}=&-\frac12\\
  \vec r=&\begin{pmatrix}
    -\cos{\omega_0t}\exp\paren{-\dfrac{\Gamma t}{2}}\\
    \sin{\omega_0t}\exp\paren{-\dfrac{\Gamma t}{2}}\\
    \ue^{-\Gamma t}-1
  \end{pmatrix}
  \intertext{For $|e\rangle$}
  \rho_{ee}=&1\\
  \rho_{ge}=&0\\
  \vec r=&\begin{pmatrix}
    0\\
    0\\
    2\ue^{-\Gamma t}-1
  \end{pmatrix}
}
Plotting the first and the second one, (the third one is just a line connecting the north and south pole)
\begin{center}
  \begin{pspicture}(-6,-6)(6,6)
    \psset{viewpoint=50 20 20 rtp2xyz,
      Decran=50,lightsrc=50 20 20,lightintensity=1}
    \psSolid[
    object=sphere,
    r=5,fillcolor=black!0,
    grid,
    incolor=black!1,
    hollow=true,
    ngrid=45 45]

    \axesIIID[linewidth=1pt,linecolor=gray,arrowsize=5pt,
    arrowinset=0,axisnames={{},{},{}},
    labelsep=10pt]
    (5,5,5)(6,6,6)
    \axesIIID[linewidth=1pt,linecolor=gray,arrowsize=0,
    arrowinset=0,axisnames={{},{},{}},
    labelsep=10pt]
    (-5,-5,-5)(-6,-6,-6)
    \defFunction[algebraic]{f1}(t){5*cos(4*t)*e^(-t/2)}{-5*sin(4*t)*e^(-t/2)}{5*e^(-t)-5}
    \psSolid[object=courbe, r=0,
    range=0 10,
    linecolor=blue,linewidth=0.03,
    resolution=1000,
    function=f1]%
    \defFunction[algebraic]{f2}(t){-5*cos(4*t)*e^(-t/2)}{5*sin(4*t)*e^(-t/2)}{5*e^(-t)-5}
    \psSolid[object=courbe, r=0,
    range=0 10,
    linecolor=red,linewidth=0.03,
    resolution=1000,
    function=f2]%
  \end{pspicture}
\end{center}

\subsection{}
In the limit of $\Delta t\rightarrow 0$ the atom haven't got enough time ($\delta^{-1}$) to figure out that it is actually detuned from the cavity resonance yet.

\section{}
\subsection{}
\eqar{
  \diff{}{t}|\psi\rangle=&\paren{\dot a+\ui\omega_1a}\ue^{\ui\omega_1t}|g\rangle+\paren{\dot b+\ui\omega_2b}\ue^{\ui\omega_2t}|e\rangle\\
  \frac{H}{\hbar}|\psi\rangle=&-a\frac{\omega_0}{2}\ue^{\ui\omega_1t}|g\rangle+b\frac{\omega_0}{2}\ue^{\ui\omega_2t}|e\rangle+\frac{\Omega_1}{2}\paren{\ue^{\ui\omega_Lt}b\ue^{\ui\omega_2t}|g\rangle+\ue^{-\ui\omega_Lt}a\ue^{\ui\omega_1t}|e\rangle}
  \intertext{If $\ue^{-\ui\omega_Lt}\ue^{\ui\omega_1t}=\ue^{\ui\omega_2t}$ (and for convinience having $\omega_1+\omega_2=0$)}
  \diff{}{t}|\psi\rangle=&\paren{\dot a+\ui\frac{\omega_L}{2}a}\ue^{\ui\omega_Lt/2}|g\rangle+\paren{\dot b-\ui\frac{\omega_L}{2}b}\ue^{-\ui\omega_Lt/2}|e\rangle\\
  \frac{H}{\hbar}|\psi\rangle=&-a\frac{\omega_0}{2}\ue^{\ui\omega_Lt/2}|g\rangle+b\frac{\omega_0}{2}\ue^{-\ui\omega_Lt/2}|e\rangle+\frac{\Omega_1}{2}\paren{b\ue^{\ui\omega_Lt/2}|g\rangle+a\ue^{-\ui\omega_Lt/2}|e\rangle}\\
  =&\frac{b\Omega_1-a\omega_0}{2}\ue^{\ui\omega_Lt/2}|g\rangle+\frac{a\Omega_1+b\omega_0}{2}\ue^{-\ui\omega_Lt/2}|e\rangle
  \intertext{Since $\ui\ddiff{}{t}|\psi\rangle=\dfrac{H}{\hbar}|\psi\rangle$}
  \ui\dot a=&\frac{b\Omega_1+a\delta}{2}\\
  \ui\dot b=&\frac{a\Omega_1-b\delta}{2}
}
\subsection{}
\eqar{
  H'=&\frac{\hbar}{2}\paren{\delta\sigma_z+\Omega_1\sigma_x}
}
\subsection{}
\eqar{
  H'=&\frac{\hbar\Omega}{2}\paren{\cos2\theta\sigma_z+\sin2\theta\sigma_x}
}
where $\sin2\theta=\dfrac{\Omega_1}{\Omega}$ and $\cos2\theta=\dfrac{\delta}{\Omega}$
\subsection{}
Since the Hamiltonian is proportional to $\vec r\cdot\vec\sigma$ where $\vec r=\cos2\theta\hat z+\sin2\theta\hat x$, the eigenvalues are $\pm\dfrac{\hbar\Omega}{2}$ with eigenvectors
\eqar{
  |+\rangle=&\cos\theta|e'\rangle+\sin\theta|g'\rangle\\
  |-\rangle=&-\sin\theta|e'\rangle+\cos\theta|g'\rangle\\
  H|+\rangle=&\frac{\hbar\Omega}{2}\paren{\cos2\theta\cos\theta|e'\rangle-\cos2\theta\sin\theta|g'\rangle+\sin2\theta\cos\theta|g'\rangle+\sin2\theta\sin\theta|e'\rangle}\\
  =&\frac{\hbar\Omega}{2}|+\rangle\\
  H|-\rangle=&-\frac{\hbar\Omega}{2}|-\rangle
}

\subsection{}
The generic solution is
\eqar{
  |\psi\rangle=&\paren{a\ue^{-\ui\Omega t/2}\cos\theta-b\ue^{\ui\Omega t/2}\sin\theta}\ue^{-\ui\omega_Lt/2}|e\rangle+\paren{a\ue^{-\ui\Omega t/2}\sin\theta+b\ue^{\ui\Omega t/2}\cos\theta}\ue^{\ui\omega_Lt/2}|g\rangle
}

\end{document}
