\documentclass[10pt,fleqn]{article}
\newcommand{\name}[1]{\def\psettitlename{#1}}
\newcommand{\course}[1]{\def\psettitlecourse{#1}}
\newcommand{\rsection}[1]{\def\psettitlersection{#1}}
\newcommand{\psetnum}[1]{\def\psettitlepsetnum{#1}}
%\usepackage[journal=rsc]{chemstyle}
%\usepackage{mhchem}
\usepackage{amsmath}
\usepackage{amssymb}
\usepackage{amsfonts}
\usepackage{esint}
\usepackage{bbm}
\usepackage{amscd}
\usepackage{picinpar}
\usepackage[pdftex]{graphicx}
\usepackage{tikz}
\usepackage{indentfirst}
\usepackage{wrapfig}
\usepackage{units}
\usepackage{textcomp}
\usepackage[utf8x]{inputenc}
% \usepackage{feyn}
\usepackage{feynmp}
\usetikzlibrary{
  arrows,
  calc,
  decorations.pathmorphing,
  decorations.pathreplacing,
  decorations.markings,
  fadings,
  positioning,
  shapes
}

\DeclareGraphicsRule{*}{mps}{*}{}
\newcommand{\ud}{\mathrm{d}}
\newcommand{\ue}{\mathrm{e}}
\newcommand{\ui}{\mathrm{i}}
\newcommand{\res}{\mathrm{Res}}
\newcommand{\Tr}{\mathrm{Tr}}
\newcommand{\dsum}{\displaystyle\sum}
\newcommand{\dprod}{\displaystyle\prod}
\newcommand{\dlim}{\displaystyle\lim}
\newcommand{\dint}{\displaystyle\int}
\newcommand{\fsno}[1]{{\!\not\!{#1}}}
\newcommand{\eqar}[1]
{
  \begin{align*}
    #1
  \end{align*}
}
\newcommand{\texp}[2]{\ensuremath{{#1}\times10^{#2}}}
\newcommand{\dexp}[2]{\ensuremath{{#1}\cdot10^{#2}}}
\newcommand{\eval}[2]{{\left.{#1}\right|_{#2}}}
\newcommand{\paren}[1]{{\left({#1}\right)}}
\newcommand{\lparen}[1]{{\left({#1}\right.}}
\newcommand{\rparen}[1]{{\left.{#1}\right)}}
\newcommand{\abs}[1]{{\left|{#1}\right|}}
\newcommand{\sqr}[1]{{\left[{#1}\right]}}
\newcommand{\crly}[1]{{\left\{{#1}\right\}}}
\newcommand{\angl}[1]{{\left\langle{#1}\right\rangle}}
\newcommand{\tpdiff}[4][{}]{{\paren{\frac{\partial^{#1} {#2}}{\partial {#3}{}^{#1}}}_{#4}}}
\newcommand{\tpsdiff}[4][{}]{{\paren{\frac{\partial^{#1}}{\partial {#3}{}^{#1}}{#2}}_{#4}}}
\newcommand{\pdiff}[3][{}]{{\frac{\partial^{#1} {#2}}{\partial {#3}{}^{#1}}}}
\newcommand{\diff}[3][{}]{{\frac{\ud^{#1} {#2}}{\ud {#3}{}^{#1}}}}
\newcommand{\psdiff}[3][{}]{{\frac{\partial^{#1}}{\partial {#3}{}^{#1}} {#2}}}
\newcommand{\sdiff}[3][{}]{{\frac{\ud^{#1}}{\ud {#3}{}^{#1}} {#2}}}
\newcommand{\tpddiff}[4][{}]{{\left(\dfrac{\partial^{#1} {#2}}{\partial {#3}{}^{#1}}\right)_{#4}}}
\newcommand{\tpsddiff}[4][{}]{{\paren{\dfrac{\partial^{#1}}{\partial {#3}{}^{#1}}{#2}}_{#4}}}
\newcommand{\pddiff}[3][{}]{{\dfrac{\partial^{#1} {#2}}{\partial {#3}{}^{#1}}}}
\newcommand{\ddiff}[3][{}]{{\dfrac{\ud^{#1} {#2}}{\ud {#3}{}^{#1}}}}
\newcommand{\psddiff}[3][{}]{{\frac{\partial^{#1}}{\partial{}^{#1} {#3}} {#2}}}
\newcommand{\sddiff}[3][{}]{{\frac{\ud^{#1}}{\ud {#3}{}^{#1}} {#2}}}
\usepackage{fancyhdr}
\usepackage{multirow}
\usepackage{fontenc}
%\usepackage{tipa}
\usepackage{ulem}
\usepackage{color}
\usepackage{cancel}
\newcommand{\hcancel}[2][black]{\setbox0=\hbox{#2}%
\rlap{\raisebox{.45\ht0}{\textcolor{#1}{\rule{\wd0}{1pt}}}}#2}
\pagestyle{fancy}
\setlength{\headheight}{67pt}
\fancyhead{}
\fancyfoot{}
\fancyfoot[C]{\thepage}
\fancyhead[R]
{
\psettitlename \\
\psettitlecourse{} Problem Set \psettitlepsetnum \\
\ifx\psettitlersection\empty
\else
Recitation Section \psettitlersection
\fi
}
\renewcommand{\footruleskip}{0pt}
\renewcommand{\headrulewidth}{0.4pt}
\renewcommand{\footrulewidth}{0pt}
\addtolength{\hoffset}{-1.3cm}
\addtolength{\voffset}{-2cm}
\addtolength{\textwidth}{3cm}
\addtolength{\textheight}{2.5cm}
\renewcommand{\footskip}{10pt}
\setlength{\headwidth}{\textwidth}
\setlength{\headsep}{20pt}
\setlength{\marginparwidth}{0pt}
\parindent=0pt
\psetnum{6}
\course{8.422}
\rsection{1}
\name{Yichao Yu}
\renewcommand{\thesection}{\arabic{section}.}
\renewcommand{\thesubsection}{(\alph{subsection})}
\renewcommand{\thesubsubsection}{\roman{subsubsection}.}

\begin{document}
\section{}
\subsection{}
For atoms that are closed enough, the interaction between them can be treated as dipole-dipole interaction (effectively integrate out the photon field). To second order of this effective Hamiltonian, the (non-degenerate) second order perturbation theory gives
\eqar{
  \Delta E\propto&\frac{d_a^2d_b^2}{R^6\paren{E_i^{(a)}+E_g^{(b)}-E_i^{(b)}-E_g^{(a)}}}
}
which scale with $R^{-6}$. The same is true for $|g_ai_b\rangle$. The perturbation theory breaks down when $E_i^{(a)}+E_g^{(b)}-E_i^{(b)}-E_g^{(a)}=0$.
\subsection{}
When $|g_ai_b\rangle$ and $|i_ag_b\rangle$ are degenerate, the energy shift become first order
\eqar{
  \Delta E\propto&\frac{d^2}{R^3}
}
\subsection{}
\eqar{
  \Gamma\propto&\omega^3d^2\\
  \frac{\Gamma}{\Delta E}=&\omega^3
}

\section{}
\eqar{
  F_c=&-\frac{\hbar c\pi^2}{240a^2}\\
  F_e=&\frac{e^2}{16\pi\varepsilon_0 a^2}\\
  \frac{\hbar c\pi^2}{240}=&\frac{e^2}{16\pi\varepsilon_0}\\
  \alpha=&\frac{e^2}{4\pi\varepsilon_0\hbar c}\\
  =&\frac{\pi^2}{60}
}

\section{}
\subsection{}
\eqar{
  \rho\paren{\theta}=&\begin{pmatrix}
    a&b\ue^{\ui\theta}\\
    c\ue^{-\ui\theta}&d
  \end{pmatrix}\\
  \angl{\rho}=&\int_{-\infty}^\infty\ud\theta
  \frac{1}{\sqrt{4\pi\lambda t}}
  \begin{pmatrix}
    a&b\ue^{\ui\theta}\\
    c\ue^{-\ui\theta}&d
  \end{pmatrix}\exp\paren{-\frac{\theta^2}{2\lambda t}}\\
  =&\begin{pmatrix}
    a&\dfrac{b}{\sqrt{4\pi\lambda t}}\dint_{-\infty}^\infty\ud\theta\exp\paren{\ui\theta-\dfrac{\theta^2}{2\lambda t}}\\
    \dfrac{c}{\sqrt{4\pi\lambda t}}\dint_{-\infty}^\infty\ud\theta\exp\paren{-\ui\theta-\dfrac{\theta^2}{2\lambda t}}&d
  \end{pmatrix}\\
  =&\begin{pmatrix}
    a&\dfrac{b}{\sqrt{4\pi\lambda t}}\dint_{-\infty}^\infty\ud\theta\exp\paren{
      -\dfrac{\paren{\theta-\ui\lambda t}^2+\paren{\lambda t}^2}{2\lambda t}
    }\\
    \dfrac{c}{\sqrt{4\pi\lambda t}}\dint_{-\infty}^\infty\ud\theta\exp\paren{
      -\dfrac{\paren{\theta+\ui\lambda t}^2+\paren{\lambda t}^2}{2\lambda t}
    }&d
  \end{pmatrix}\\
  =&\begin{pmatrix}
    a&b\exp\paren{-\dfrac{\lambda t}{2}}\\
    c\exp\paren{-\dfrac{\lambda t}{2}}&d
  \end{pmatrix}\\
}
\subsection{}
Since the Hamiltonian does nothing to $|g\rangle$, $|g0\rangle$ will remain the same. For $|e*\rangle$, the environment will undergo Rabi flopping. Also assume $\gamma$ is real since the phase of it can be absorbed in $|1\rangle$. The evolution of state,
\eqar{
  |\psi\paren{t}\rangle=&a|g0\rangle+b\paren{\cos\frac{\gamma t}{\hbar}|e0\rangle+\ui\sin\frac{\gamma t}{\hbar}|e1\rangle}
  \intertext{Atomic density matrix}
  \rho=&\begin{pmatrix}
    \abs{a}^2&ab^*\cos\dfrac{\gamma t}{\hbar}\\
    a^*b\cos\dfrac{\gamma t}{\hbar}&\abs{b}^2
  \end{pmatrix}
}
\subsection{}
At time $t$ the average density matrix is
\eqar{
  \angl{\rho}=&\frac{1+\ue^{-\lambda t}}{2}\begin{pmatrix}
    a&b\\
    c&d
  \end{pmatrix}+\frac{1-\ue^{-\lambda t}}{2}\begin{pmatrix}
    a&-b\\
    -c&d
  \end{pmatrix}\\
  =&\begin{pmatrix}
    a&b\ue^{-\lambda t}\\
    c\ue^{-\lambda t}&d
  \end{pmatrix}
}

\end{document}
