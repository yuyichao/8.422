\documentclass[10pt,fleqn]{article}
\newcommand{\name}[1]{\def\psettitlename{#1}}
\newcommand{\course}[1]{\def\psettitlecourse{#1}}
\newcommand{\rsection}[1]{\def\psettitlersection{#1}}
\newcommand{\psetnum}[1]{\def\psettitlepsetnum{#1}}
%\usepackage[journal=rsc]{chemstyle}
%\usepackage{mhchem}
\usepackage{amsmath}
\usepackage{amssymb}
\usepackage{amsfonts}
\usepackage{esint}
\usepackage{bbm}
\usepackage{amscd}
\usepackage{picinpar}
\usepackage[pdftex]{graphicx}
\usepackage{tikz}
\usepackage{indentfirst}
\usepackage{wrapfig}
\usepackage{units}
\usepackage{textcomp}
\usepackage[utf8x]{inputenc}
% \usepackage{feyn}
\usepackage{feynmp}
\usetikzlibrary{
  arrows,
  calc,
  decorations.pathmorphing,
  decorations.pathreplacing,
  decorations.markings,
  fadings,
  positioning,
  shapes
}

\DeclareGraphicsRule{*}{mps}{*}{}
\newcommand{\ud}{\mathrm{d}}
\newcommand{\ue}{\mathrm{e}}
\newcommand{\ui}{\mathrm{i}}
\newcommand{\res}{\mathrm{Res}}
\newcommand{\Tr}{\mathrm{Tr}}
\newcommand{\dsum}{\displaystyle\sum}
\newcommand{\dprod}{\displaystyle\prod}
\newcommand{\dlim}{\displaystyle\lim}
\newcommand{\dint}{\displaystyle\int}
\newcommand{\fsno}[1]{{\!\not\!{#1}}}
\newcommand{\eqar}[1]
{
  \begin{align*}
    #1
  \end{align*}
}
\newcommand{\texp}[2]{\ensuremath{{#1}\times10^{#2}}}
\newcommand{\dexp}[2]{\ensuremath{{#1}\cdot10^{#2}}}
\newcommand{\eval}[2]{{\left.{#1}\right|_{#2}}}
\newcommand{\paren}[1]{{\left({#1}\right)}}
\newcommand{\lparen}[1]{{\left({#1}\right.}}
\newcommand{\rparen}[1]{{\left.{#1}\right)}}
\newcommand{\abs}[1]{{\left|{#1}\right|}}
\newcommand{\sqr}[1]{{\left[{#1}\right]}}
\newcommand{\crly}[1]{{\left\{{#1}\right\}}}
\newcommand{\angl}[1]{{\left\langle{#1}\right\rangle}}
\newcommand{\tpdiff}[4][{}]{{\paren{\frac{\partial^{#1} {#2}}{\partial {#3}{}^{#1}}}_{#4}}}
\newcommand{\tpsdiff}[4][{}]{{\paren{\frac{\partial^{#1}}{\partial {#3}{}^{#1}}{#2}}_{#4}}}
\newcommand{\pdiff}[3][{}]{{\frac{\partial^{#1} {#2}}{\partial {#3}{}^{#1}}}}
\newcommand{\diff}[3][{}]{{\frac{\ud^{#1} {#2}}{\ud {#3}{}^{#1}}}}
\newcommand{\psdiff}[3][{}]{{\frac{\partial^{#1}}{\partial {#3}{}^{#1}} {#2}}}
\newcommand{\sdiff}[3][{}]{{\frac{\ud^{#1}}{\ud {#3}{}^{#1}} {#2}}}
\newcommand{\tpddiff}[4][{}]{{\left(\dfrac{\partial^{#1} {#2}}{\partial {#3}{}^{#1}}\right)_{#4}}}
\newcommand{\tpsddiff}[4][{}]{{\paren{\dfrac{\partial^{#1}}{\partial {#3}{}^{#1}}{#2}}_{#4}}}
\newcommand{\pddiff}[3][{}]{{\dfrac{\partial^{#1} {#2}}{\partial {#3}{}^{#1}}}}
\newcommand{\ddiff}[3][{}]{{\dfrac{\ud^{#1} {#2}}{\ud {#3}{}^{#1}}}}
\newcommand{\psddiff}[3][{}]{{\frac{\partial^{#1}}{\partial{}^{#1} {#3}} {#2}}}
\newcommand{\sddiff}[3][{}]{{\frac{\ud^{#1}}{\ud {#3}{}^{#1}} {#2}}}
\usepackage{fancyhdr}
\usepackage{multirow}
\usepackage{fontenc}
%\usepackage{tipa}
\usepackage{ulem}
\usepackage{color}
\usepackage{cancel}
\newcommand{\hcancel}[2][black]{\setbox0=\hbox{#2}%
\rlap{\raisebox{.45\ht0}{\textcolor{#1}{\rule{\wd0}{1pt}}}}#2}
\pagestyle{fancy}
\setlength{\headheight}{67pt}
\fancyhead{}
\fancyfoot{}
\fancyfoot[C]{\thepage}
\fancyhead[R]
{
\psettitlename \\
\psettitlecourse{} Problem Set \psettitlepsetnum \\
\ifx\psettitlersection\empty
\else
Recitation Section \psettitlersection
\fi
}
\renewcommand{\footruleskip}{0pt}
\renewcommand{\headrulewidth}{0.4pt}
\renewcommand{\footrulewidth}{0pt}
\addtolength{\hoffset}{-1.3cm}
\addtolength{\voffset}{-2cm}
\addtolength{\textwidth}{3cm}
\addtolength{\textheight}{2.5cm}
\renewcommand{\footskip}{10pt}
\setlength{\headwidth}{\textwidth}
\setlength{\headsep}{20pt}
\setlength{\marginparwidth}{0pt}
\parindent=0pt
\psetnum{5}
\course{8.422}
\rsection{1}
\name{Yichao Yu}
\renewcommand{\thesection}{\arabic{section}.}
\renewcommand{\thesubsection}{(\alph{subsection})}
\renewcommand{\thesubsubsection}{\roman{subsubsection}.}

\begin{document}
\section{}
\subsection{}
\eqar{
}
\subsection{}

\section{}
\subsection{}
For atoms in an isotropic environment
\eqar{
  \langle g|\vec d|g \rangle=&\langle i|\vec d|i \rangle\\
  =&0
}
The first order correction to the energy is $0$.\\
Second order
\eqar{
  \delta E=&-\sum_{i_a, i_b}\frac{\abs{\langle i_ai_b|H_{el}|gg\rangle}^2}{\hbar\omega^a_{ig}+\hbar\omega^b_{ig}}\\
  \langle i_ai_b|H_{el}|gg \rangle=&\frac{e^2}{R^3}\paren{\vec r^a_{ig}\cdot\vec r^b_{ig}-3x^a_{ig}x^b_{ig}}
  \intertext{Since the system is symmetric for rotation around $\vec R$, the ``good'' states to calculate the energy shift should also have the same symmetry. Therefore $\vec r^a$ and $\vec r^b$ should be along the $\vec R$ ($x$) direction}
  \langle i_ai_b|H_{el}|gg \rangle=&-\frac{2e^2x^a_{ig}x^b_{ig}}{R^3}\\
  \delta E=&-\sum_{i_a, i_b}\frac{1}{\hbar\omega^a_{ig}+\hbar\omega^b_{ig}}\abs{\frac{2e^2x^a_{ig}x^b_{ig}}{R^3}}^2\\
  =&-\frac{4e^4}{\hbar R^6}\sum_{i_a, i_b}\frac{\abs{x^a_{ig}}^2\abs{x^b_{ig}}^2}{\omega^a_{ig}+\omega^b_{ig}}
}
\subsection{}
\eqar{
  \abs{x_{ig}}^2=&\frac{\hbar f_{ig}}{2m\omega_{ig}}\\
  \delta E=&-\frac{4e^4}{\hbar R^6}\sum_{i_a, i_b}\frac{1}{\omega^a_{ig}+\omega^b_{ig}}\frac{\hbar f^a_{ig}}{2m\omega^a_{ig}}\frac{\hbar f^b_{ig}}{2m\omega^b_{ig}}\\
  =&-\frac{\hbar e^4}{m^2R^6}\sum_{i_a, i_b}\frac{f^a_{ig}f^b_{ig}}{\paren{\omega^a_{ig}+\omega^b_{ig}}\omega^a_{ig}\omega^b_{ig}}
}
\subsection{}
Since the oscillator strength for the ground state is always positive, if the first excited state has $f\approx1$, the contribution from other transitions are small. With this approximation,
\eqar{
  \abs{x_{ig}}^2=&\frac{\hbar\omega_{ig}}{2e^2}\alpha_g\\
  \delta E=&-\frac{4e^4}{\hbar R^6}\frac{1}{\omega^a_{ig}+\omega^b_{ig}}\frac{\hbar\omega^a_{ig}}{2e^2}\alpha^a_g\frac{\hbar\omega^b_{ig}}{2e^2}\alpha^b_g\\
  =&-\frac{\hbar}{R^6}\frac{\omega^a_{ig}\omega^b_{ig}}{\omega^a_{ig}+\omega^b_{ig}}\alpha^a_g\alpha^b_g\\
  C_6=&\hbar\frac{\omega^a_{ig}\omega^b_{ig}}{\omega^a_{ig}+\omega^b_{ig}}\alpha^a_g\alpha^b_g
}

\end{document}
